\documentclass[a4paper]{book}
\usepackage{hyperref}
\usepackage[utf8]{inputenc}
\usepackage{todonotes}
\usepackage{censor}
\usepackage{enumitem}
\usepackage[breakable, skins]{tcolorbox}

\usepackage[a4paper]{geometry}
\usepackage{multicol}
\usepackage{tocloft}
   
\newcommand{\quoteauthor}[1]{\begin{flushright}--\textit{#1}\end{flushright}}

\newenvironment{goddesc}{
\begin{tcolorbox}
\begin{description}[style=multiline,leftmargin=3cm]
\newcommand{\Domains}[1]{\item[Domains] ##1}
\newcommand{\Symbol}[1]{\item[Symbol] ##1}
\newcommand{\Level}[1]{\item[Level] ##1}
}{
\end{description}
\end{tcolorbox}
}

\newcommand{\VariantRule}[1]{
\addcontentsline{toc}{}{\protect\hspace{5em}\textit{Variant Rule: #1}\protect\cftdotfill{\cftdotsep} }%
\subsubsection*{Variant Rule: #1}\setcounter{tocdepth}{1}}%
\newcommand{\rolltablecolumnnumber}{1}

\newenvironment{rolltable}[2][\textheight]{
\begin{tcolorbox}[enhanced jigsaw, breakable, break at=#1]
\begin{center}
#2
\end{center}
\begin{footnotesize}
\begin{description}[style=multiline,leftmargin=1cm]
}{
\end{description}
\end{footnotesize}
\end{tcolorbox}
}


\definecolor{adventureideabg}{HTML}{aceca1}
\definecolor{adventureideaheaderbg}{HTML}{629460}

\newenvironment{itquote}
  {\begin{quote}\itshape}
  {\end{quote}\ignorespacesafterend}

\newenvironment{itpars}
  {\par\itshape}
  {\par}
  
\newenvironment{adventureidea}[2][\textheight]
  {\begin{tcolorbox}[title=Adventure Idea: #2,colback=adventureideabg,colbacktitle=adventureideaheaderbg, breakable, break at=#1]}
  {\end{tcolorbox}}


\hypersetup{
    pdftex,
    colorlinks,
    citecolor=black,
    filecolor=black,
    linkcolor=black,
    urlcolor=blackf
}
\title{Spirals}
\author{Anthony \censor{Super}}

\begin{document}
\setcounter{tocdepth}{1}
\maketitle



Spirals is a Dungeons and Dragons 5\textsuperscript{th} edition campaign setting.
It is a highly magical world with unusual terrain and plot elements.
This is not a module, but rather a campaign guidebook---it provides a description of a world and those who live within it, so you can make your own adventures.
\listoftodos
\tableofcontents 

\part{The World's Structure}
\chapter{Tendrils}

\begin{quote}
\itshape
Alysha the Warlock takes the moment to rest, laying down in the grass and looking up.
Above her, land swirls through the sky, like passageways in a great ant colony.
Over the course of her short life, she has walked, ridden, and fled upon many.
Yet, now, with their vastness above her, she feels just as she did back in Ryngolang, a young elf helping her mother do laundry and dreaming of the world beyond.
\end{quote}

Spirals is perhaps most unique for how its land is organized.
Spirals does not take place on a planet, or even several planets.
In fact, planets in the universe are a rarity.
Instead, matter is organized into twisting, irregular shapes, moving throughout reality in an infinite variety of different ways.
These shapes are commonly referred to as \textit{Tendrils}.

The tendrils are not a completely unorganized mess.
There are six main tendrils, the arteries of the world. They start at a central location, called \textit{The Core}, and grow outwards, growing in width as they travel apart.

Those tendrils eventually branch off into many more, very shortly after leaving the core.
These branches themselves have branches, creating a fractal pattern of land that criss-crosses the sky like a giant spider web.

This is not to say that the tendrils are a perfect mesh network.
Some are not attached to anything, floating alone in space.
Some twist back on themselves, or spin off in random fractal directions.
Some may even take the form of floating spheres, cubes, or other solid polygons.


Tendrils are a fairly unique concept.
In this chapter, we discuss how they work, how to make your own, and other topics you need to run your game.

\begin{multicols*}{2}
\section{Physics}
The Tendrils that make up the land (and sea) in Spirals are great masses of material floating in a magical aether.
They are made up of many materials, ranging from solid rock to loose dirt to water.
Some tendrils are even made entirely of air or other gasses, although those are obviously harder to detect.

The tendrils have gravity, but it works a bit differently than it does in real life.
The gravity of a tendril is primarily a function of the magic keeping it up.
As a result, some tendrils can be tubes as little as ten feet around, but still have gravity to hold large objects or creatures safely on their surface.
Tendrils which are close to each other may have gravitational fields so small that one can hop from one to the other, naturally finding their feet pulled to the floor as they transition.

Tendrils have a wide range of sizes. 
Some of the smaller ones can be thin pillars of earth that a typical human could wrap their hand around, while others can be so thick they take years to encircle on foot.

\subsection*{Lighting}
Tendrils experience a day and night cycle, just like in reality.
Unlike in reality, however, the light generally does not have a defined source.
Lighting and heat are a function of the same magic that holds up the tendrils, and patches of day move across them in strips several thousand miles wide.
Each strip takes around sixteen hours to move across a tendril during the warmer months, with a "night" lasting around six hours.
During the colder months, the night can last up to twelve hours, with day taking as few as five.
Different tendrils may have different timings.

\subsection*{Tendril Collapse}
The tendrils of Spirals are, in the general case, incredibly stable.
Physical forces seem to have no effect on the actual position of the landmasses, and even the most powerful of explosions will, at most, leave a large crater or obliterate a chunk of a thinner spiral, leaving both ends intact.
Under rare circumstances, however, a tendril can collapse completely.
Typical causes for this are:
\begin{description}
\item[Intentional Destruction:] groups of extremely powerful wizards may be able to collapse a tendril entirely by unwinding the magical framework in which it sits. 
This is occasionally done by governments to non-inhabited tendrils, in order to make trade routes easier, or (occasionally) to gain the raw materials that make up the Tendril.
Doing this costs an extremely large amount of money, as it requires the equivalent of several dozen \textit{wish} spells.
\item[Extreme Magical Disturbance:] certain magical acts may cause the magic holding a tendril up to unravel on its own, destroying a Tendril.
This happens so incredibly rarely that it has only happened in four cases that were well-documented.
\item[Acts of the gods:] the more powerful gods may chose to collapse a tendril to punish mortals. It should be noted, however, that collapsing a tendril with people on it is one of the most extreme actions a god can take. 
Even chaotic evil gods may think twice before doing so---especially because doing so is a great way to unite their enemies against them.
\item[Natural Decay:] some tendrils simply lose power over time, and will collapse on their own. This process, however, takes quite a long time, and those who are remotely magically sensitive will easily be able to detect what is happening a year or more before it does.
That is not to say that natural collapse never causes issues---a year is a rather short time period to move a city, much less an entire empire.
\end{description}

\begin{adventureidea}{Tendril Collapse}
Tendril collapse can provide an interesting plot hook for your own adventures.
Perhaps a crazed villain decides to destroy a tendril for his own material gain, or simply because he does not like it.
Perhaps the players must convince a group of wizards to help them collapse a spiral that houses an evil dragon or tarrasque, before it wakes up.
Perhaps a spiral has suddenly collapsed, killing thousands and leaving the survivors pointing fingers at their rival groups, looking for somebody to blame.
\end{adventureidea}


\section{Creating a Tendril}
The world of Spirals is home to uncountable tendrils, making their way across the stars.
Some of the major ones are described in this document.
However, the situation may arise in your campaigns where you wish to throw your players onto a brand-new tendril, that you created, or when you want a unique Tendril for a character backstory.
Here, we will provide some general guidelines for doing so, as well as roll tables to help you if you get suck (or if you just like the element of randomness).

\subsubsection*{Few Limitations}
The nature of tendrils gives you extreme freedom as a DM.
In fact, a tendril is able to have essentially any shape, size, environment, or other characteristic that you desire.

As such, the limitations mainly exist in terms of \textit{narrative consistency}.
A tendril should not logically break the society of Spirals.
For example, a tendril made of solid gold that's thousands of miles long would reduce the value of gold to near-worthlessness, due to the laws of supply and demand.
Putting such a tendril in your campaign may not be a good idea---unless you can justify it.
For example, perhaps the surface of the tendril is completely impenetrable, mocking those who attempt to mine it.
Maybe it is extremely far away from any settled area, accessible only via a long and near-suicidal journey through a tendril that alternates between bitter ice and scorching desert, and your players are simply the first to reach it.
Perhaps everybody knows it exists, but it is floating by itself far from any other landmass, through a vacuum that is unaccessible for even flying creatures, with an aura around it that prevents magical teleportation.\footnote{Such a tendril is almost guaranteed to be irresistible to your players, likely eventually resulting in a spaceship mission to acquire its wealth. Of course, a space race campaign seems to have a lot of potential, so perhaps that's a good thing.}

In truth, almost nothing is completely off-limits.
If you have an idea for an especially unique Tendril, just be sure to think through its logical consequences.
You may even want to run your idea past others, to get a second opinion on the matter.


\subsection{Location}
Tendrils branch off from one another, creating a tree with infinitely many branches.\footnote{Assuming you are not using the formal, mathematical definition of a tree. Technically, in that case, they make a multigraph (if you consider cities to be nodes)}.


\begin{rolltable}{Tendril Locations (d12)}
\item[1-3] Offshoot of the root tendril most similar in climate, near the core
\item[4-5] Offshoot of the root tendril most similar in climate, far downtendril
\item[6]  Offshoot of the root tendril most different in climate, near the core
\item[7] Offshoot of the root tendril most different in climate, far downtendril
\item[8-10] Offshoot of a random root tendril 
\item[11] Offshoot of a minor tendril
\item[12] Floating alone in space, accessible only by a portal
\end{rolltable}


\subsection{Shape}
Tendrils come in many shapes.
Depending on the story you want to tell, these shapes may be vitally important to the plot, or mostly incidental.
If you simply want a jungle adventure, for example, a jungle Tendril of any shape will probably suffice.
However, this is not always the case.
Tendrils of different shapes can be used to affect gameplay and story significantly.

Let's think of a Tendril which is a rectangular prism.
People on one side of the tendril will have severe difficulty seeing other people (or creatures) that are on the other sides of the tendril, even if those people may be dangerously close to them.
Your players (or antagonists) may be able to use this to their advantage, in order to hide from a dangerous monster or to set up an ambush.


\begin{rolltable}{Tendril Shapes (d12)}
\item[1-2] Cylinder
\item[3-4] Rectangular Prism
\item[5-6] Hollow cylinder, with walkable inside surfaces on both sides
\item[7-8] Half pipe
\item [9-10] Flat surface
\item[11] Planetoid
\item[12] Floating chunks of flat land
\end{rolltable}

\subsection{Climate}
Tendrils vary widely in climate, from dense jungle to arid desert.
Some are even composed entirely of water.

Tendrils tend to be fairly consistent in climate across their surface, although some do change.
Tendrils that suddenly undergo an extreme transition (such as from tundra to desert)  may be considered to actually be two tendrils pressed together.

\begin{rolltable}[0.5\textheight]{Tendril Climates (d10)}
\item[1] Underwater
\item[2] Jungle
\item[3] Desert
\item[4] Forest
\item[5]  Swamp
\item[6] Tundra
\item[7] Plains
\item[8] Ice
\item[9] Rocky
\item[10] Volcanic
\end{rolltable}

\subsection{Complications}
Tendrils are landmasses, but they are magical landmasses.
As such, they can occasionally have certain \textit{unique} features.

These features can be mundane or highly exciting, useful or hindering, logical or unusual.
We recommend that you give Tendrils complications sparingly, so that the ones which do have them are special.

\begin{rolltable}{Tendril Complication Type (d6)}
\item[1] Negative Complication
\item[2-5] No complication
\item[6] Positive complication
\end{rolltable}

\todo{Finish this roll table}
\begin{rolltable}{Negative Tendril Complications (d12)}

\item[1-2] The tendril is highly infested with monsters at a much greater rate than normal.
\item[3-4] The tendril is cursed, and nothing edible to sentient races can grow on it.
\item[5-6] The tendril suffers from extreme and frequent Earthquakes.
Every minute, roll a d20.
On a 1, an Earthquake happens, and all creatures on the ground must make a DC 15 DEX save to keep their footing.
\item[7-8] The tendril has a high amount of volcanic activity, and lava oozes up from cracks frequently.
\item[9-10] \textbf{Need this}
\item[11] \textbf{And this}
\item[12] Roll twice on this table, and use both complications.
\end{rolltable}



\todo{Finish this as well}
\begin{rolltable}{Positive Tendril Complications (d12)}
\item[1-2] The tendril is extremely peaceful, and rest comes easily.
Only half of the usual time is required to complete long and short rests.
\item[3-4] The tendril is bountiful.
All checks to forage automatically succeed.
\item[5-6] This tendril is blessed by a good god.
Pick a god at random.
Any checks related to something within this god's domain have advantage.

\item[12] Roll twice on this table, and use both complications.

\end{rolltable}

\end{multicols*}
\chapter{Magic}
\label{chapter:magic}

Magic in Spirals, on the surface, is similar to magic in other worlds.
Mechanically it works much the same: it may be granted in a variety of ways, channeled into spells, and brewed into potions.
Beneath the surface, however, the way the world of Spirals interacts with its magic is quite different.

\begin{multicols}{2}
\section{Presence}
In Spirals, the presence of magic is felt in the lives of all creatures.
It is what holds the Tendrils together, gives them form and substance, lights them during the day, and, occasionally, creates them from nowhere.


\section{Arcane Magic}
When an arcane caster uses a spell in Spirals, they seek to move the ambient magic around them, to give it form and substance.
This is a dangerous task indeed---magic, at its core, \textit{desires} to be used, to make its mark on the world, and it eagerly rushes to a caster, attempting to make that desire reality.
Unfortunately for the caster, however, the eagerness of the magic to be used does not extend to a desire to be controlled, or to be limited.
Without careful focus and application of arcane techniques, the caster will lose control of her magic, and it will run wild within her.
This can have a wide variety of effects, but, unfortunately, the most common by far is to cause the caster to burst into flames.

Those who use arcane magic must study carefully, and keep their minds clear.

\subsection{Enchanting}
Spirals has a lot of ambient magic, and clever wizards can tap into it, creating enchanted objects.
For simple enchantments this is not difficult.
If one wishes, for example, they could easily direct ambient magic to propel a vehicle or heat a bit of metal.
These simple energy transfer spells are reliable, and, as a result, vehicles like landcycles and magical forges are somewhat common.

Spells which require higher levels of precision, however, are much harder to contain within an object.
If one wished to create a sword which cut through enemies more efficiently, for example, they would need to take care to temper the enchantment so it draws only the required amount of energy from the surrounding area.
Failure, of course, would result in the sword exploding violently the moment the enchantment was cast.
As a result, enchanted weapons and items are rarer than in other settings, as they are more difficult to make.
It should be noted, however, that the high levels of ambient magic increases the overall potential for enchanted items.
Few, if any, wizards have ever lived who were skilled enough to take advantage of this potential, but, if they could, the results could be wondrous (or devastating) indeed.

\subsection{Magical Creatures}
The greater presence of magic results in a greater presence of magical creatures, as well.
Dragons in Spirals are still relatively rare, for example, but there are still enough of them that almost all the larger armies are able to contract a few for special operations work.
Horrifying monsters such as mind flayers or beholders, too, are in no short supply.
Giant animals, too, are fairly commonplace.

This seems like it may be cataclysmically disastrous for mortal civilization, but, in truth, it's not as bad as you'd think.
The magic that allows the world to be filled with wyverns and Slaads also brings up powerful heroes who can deal with such things.

\end{multicols}


\section{Variant Rule: Extremely Wild Magic}
The naturally wild nature of magic in Spirals makes being a sorcerer even more taxing.
Those who use wild magic may occasionally experience effects even more random than normal.

Using this variant rule, wild magic sorcerers get a very small chance to lose control of their magic in a \textit{ridiculous} way.
Whenever you roll a 1 on the wild magic dice, an additional D20, and a D8.\footnote{If you rarely have a wild magic sorcerer roll a wild magic dice, you may want to remove the d8 from the equation.}
If both are one, roll on the extremely wild magic table to see what sort of effect manifests.

Be warned that the effects of this rule will make your campaign quite silly.
Many will grant your sorcerer or one of his allies a new ability, permanently effect their appearance, or otherwise destroy any sense of narrative cohesion on your campaign.
The chances of getting to roll on the Supremely Wild Magic table are $ \frac{1}{3200} $, but the effects are potentially cataclysmic, and could derail your entire campaign in a glorious way.

\todo[inline]{Finish and edit this roll table}
\begin{rolltable}[0.9\textheight/0pt]{Supremely Wild Magic}
\item[1-2] Roll on this table at the start of your turn for the next two minutes.
If you get this item again, the duration is refreshed.
Duration can be refreshed a maximum of twice.
If your roll lands on this, strap in.
You're about to see some serious shit.
\item[3-4] Gain a new class ability, \textit{Ultimate Perception.}
This ability may be cast after a short rest.
Roll a d20.
On a 20, you gain vision through the eyes of every creature within a 300-foot radius of where you cast it.
The effect lasts until the creatures die, or \textbf{you} leave the radius---the creatures may leave, and you retain vision.
\item[5-6] You burst into flames. You take four times your maximum health as damage.
You can chose one target within 10 feet of you to also take this damage.
If this spell kills you (which it most likely will), you instantly and inexplicably gain a cosmically significant position in whatever your afterlife would be, of your DM's choosing.
\item[6-7] A tarrasque appears 2d100 squares away from you. 
It heads for the nearest populated area at its maximum movement seed. 
The tarrasque disappears after two hours.
\item[7-8] You (and all allies within 100 feet) instantly change race in terms of appearance, gaining the new race's bonuses, and retaining those of your old race.
The DM may decide if you get to pick the new race, or if it is randomly selected.
\item[9-10] The world wants to kill you for 1d4 days.
Every five minutes, \textit{Meteor Storm} will be cast at whatever position you were at in the previous round, until you die, or the duration of the effect ends.
\item[11-12] On all subsequent level ups, the additional hit points you gain are doubled.
\item[13-14] Your appearance changes to that of a small archdemon.
You also gain the ability to cast \textit{Finger of Death} as a bonus action once a month.
\item[15-16] You gain \textit{hordelust.}
You will become physically sick if you do not increase your net worth by at least 5\% every week, gaining disadvantage on WIS and DEX saving throws if you fail to do so.
You gain advantage on all checks relating to the appraisal of wealth, as well as Persuasion checks when buying or selling items.
\item[17-18]  Roll a d4. Your height permanently increases by the result, in feet.
\item[19-20] The heat around you spontaneously vanishes.
All creatures in a 30-foot radius of you take 10d10 ice damage. 
If a creature dies as a result of this roll, their body becomes an ice sculpture.
\item[21-22] You feel the power of sacrifice within you.
You gain the ability to heal up to three creatures to full whenever you reduce yourself to negative hit points.
When you perform this sacrifice, you lose the ability to be magically healed until you are non-magically restored to consciousness.
Recharges on a long rest.
\item[23-24] You and all creatures you can see suddenly become magically attracted to a faraway tendril. 
You will fall towards it, upwards through the air, until you touch down on its surface.
Your falling speed is capped to 20 $\frac{m}{s}$.
The fall may take several days, or even weeks, at the discretion of your DM.
\item[25-26] You permanently grow feathers all over your body, and gain the ability to speak to birds.
\item[27-28] You permanently grow a pair of wings, gaining a flying speed equal to your walking speed if unarmored, or wearing armor specially deigned to allow you to use the wings.
You also gain a wing attack. 
It has a reach of 10 feet, has 3 plus your dexterity modifier to hit, and does 1d12 damage.
You may use this attack as a bonus action.
\item[29-30] All gold and platinum within 50 feet of you permanently vanishes.
You gain the ability to cause this effect as a standard action, with a recharge time of one week.
\item[31-32] All creatures you have witnessed the death of in the last week are instantly resorted to life as if by the \textit{reincarnation} spell.
\item[33-34] Creatures automatically fail all saving throws against your spells for the next 1d4 minutes.
\item[35-36] All plants and animals within a ten-mile radius are put under the effects of the \textit{Awaken} spell, minus the charm.
\item[37-38] Your physical form loses its cohesion.
You gain the ability to change you physical appearance to any creature you have seen within one size category of yourself as a bonus action, which also changes your voice.
When not imitating another creature, you appear blurry and ethereal.
\item[39-40] Your strength permanently increases by three, but you physically gain the appearance of somebody who is severely malnourished.
\item[41-42] You gain 8 eye-stalks, like a beholder. You get a +5 to wisdom checks that rely on sight.
\item[43-44] A new Tendril sprouts out of the ground 5 feet in front of you. The DM decides its properties.
\item[45-46] Magic screams into your veins, begging you to release it.
On your turn, you must expend half of your remaining spell slots on spells of your choosing, and cast all of your cantrips once.
Each spell is cast as a free action regardless of its normal cast time.
Immediately afterwards, you will fall unconscious for 4d8+12 hours.
You gain the ability to manifest this effect once every four months.
\item[47-48] You and all creatures within 400 feet become Tyrannosaurus Rexes.
You alone keep your ability to cast spells, although everybody keeps their normal stats and intelligence.
This effect may only be reversed by the usage of a \textit{Greater Restoration} spell, although it may be difficult to convince a cleric to cast it on your party---especially since you can't speak in anything that isn't a roar.
\item[49-50] The magic around you is so eager to help you that it forgets about gravity for a moment.
You (and all other creatures you can see) become weightless for fifteen minutes.
You may manifest this effect once a month.
\item[51-52] You gain the ability to speak in tongues, which allows creatures that understand any language to understand your speech.
You also gain the ability to understand all languages.
\item[53-54] All weapons in a 100-foot radius of you are transformed into foam replicas, which do 1d2 bludgeoning damage, but keep enchantments.
This effect manifests regularly in monthly intervals.
\item[55-56] From now on, critical hits on offensive spells you cast will create a 1d8 foot deep crater below the target.
\item[57-58] You permanently gain a dragonborn breath weapon, of your choosing. If you are already a dragonborn, you get another breath weapon.
\item[59-60] Every unoccupied space on the map is filled with a flumph, which is controlled by the DM and frightened of you.
\item[61-62] Time stops in a 400-foot radius (centered on you) for a duration of one year. Nobody may enter the zone of stopped time.
\item[63-64] Magic enjoys how you kill things.
Form now on, when you deal a killing blow, you regain health equal to the damage the blow did.
\item[65-66] A huge forest sprouts around you, along with an entire civilization of very confused Elves.
The Elves are aware that they were just brought into existence, but somehow have knowledge equivalent to their age.
\item[67-68] You gain a suit of huge magical armor for one minute. 
Your size is changed to huge for the duration, and your AC increases by +10.
You may manifest this effect once a month.
\item[69-70] You become immune to all mind-alerting substances.
\item[71-72] All your skin appears to fall off, revealing the muscles and organs beneath. 
Somehow, you are unharmed.
Your skin will grow back in 1d4 weeks, however, whenever you drop to zero hit points, it will fall off again.
Medicine checks against you have advantage while skinless.
You also gain advantage on intimidation checks, owing to your horrifying appearance, with the penalty of disadvantage on persuasion checks.
\item[73-74] You gain the ability to scream incredibly loudly.
This action causes creatures within 30 feet of you to take 2d8 thunder damage, and shatters any non-magical glass item in the radius.
It recharges on a short rest.
\item[75-76] You can trade your life force for time.
You gain the ability to become immune to all damage for one minute, instantly dropping to zero hit points at the end of the duration.
\item[77-78] All people of noble blood within 40 miles disappear.
The DM decides where they go, or if they die.
Those within the world temple are immune.
\item[79-80] You gain the ability to speak to small woodland creatures.
\item[81-82] A clone of you, with all your memories and abilities, appears right next to you.
It will die in one week, a fact which it knows, and is somewhat upset about.
It may have some requests of things it wants to do before it perishes, or it may seek to get revenge, depending on your personality.
\item[83-84] The spell gets lost on the way to its target.
Somehow, it manages to hit something very far away.
\item[85-86] HELP
\item[87-88] Everyone within 200 feet of you gain a permanent psychic link to each others mind and know what each of you are thinking at all times.
\item[89-90] HELP
\item[91-92] The next 1d8+1 times you die, you instantly come back to life, as if via the \textit{Reincarnate} spell. 
\item[93-94] HELP
\item[95-96] The world desires a comedy.
As a standard action, you can chose to create a 40-foot-square \textit{Zone of Bad Luck}, which gives disadvantage on \textbf{all} rolls, centered around yourself.
The zone of bad luck follows you around, and lasts for two minutes.
You may use this ability once a month.
\item[97-98] Your maximum sorcery points increases permanently by 1d12.
You also regain all expanded sorcery points.
\item[00-00] Everybody in your party gets 2d4+1 free levels in sorcerer, including you. These levels \textbf{do not count to the required XP to level up,} and are granted regardless of stat eligibility.

\end{rolltable}

\begin{multicols}{2}

\section{Magic Derived from Nature}
Those who walk in wild places may have magic of their own.
Unlike arcane magic, it is not derived from the listless sort of magic that simply exists, but from a more purposeful, directed form: that which gives nature its shape.
This magic is found in the ground of the Tendrils, and in the features on (or below) their surface.
It is more introspective, gentler than the arcane magic, granting wondrous abilities to those it deems worthy. 

\subsection{Nature without Life}
Unlike in other settings, the magic of nature is not necessarily the magic of life---Tendrils covered in parched desert or suffocating ice have magic in equal measure as those painted with plants and trodden upon by animals.
This magic, however, rarely allows itself to be used by those among the living, preferring the inanimate.
When it does allow itself to be used, it is generally out of self-preservation: the desire to have its domain kept as it likes it, with little life.
These servants rarely cause direct harm to the living, as the rotting of flesh is itself a haven for life, choosing instead to expel the living to more suitable places.
\label{druidofdecay}
Occasionally, however, a desolate place will grow discomforted, unsatisfied with its domain.
In these cases, a Druid of Decay may arise, blindly serving his or her arcane master, singly devoted to the goal of expanding desolation and removing life.
Most often these evils are quickly dispatched by those who preserve living nature, but they occasionally may have a larger effect.

\begin{adventureidea}[0.3\textheight]{Desolation}
Druids of decay, when they arise, are a horrible plague upon the world.
They are often extremely powerful, and hell-bent on expanding the desolation of their home tendril into other, vibrant areas of life.
Worse, they often retain their sanity and ability to plan, making them dangerous foes.

A druid of decay can make a great antagonist for your campaign, especially because they can have such variance in personality.
Some are actually quite diplomatic: they'll give the populations of towns and cities a chance to retreat before destroying them, and might even intentionally scare off animals as they spread their desolation.
Others are significantly less charitable, and may intentionally cause as much death as they possibly can.
\end{adventureidea}

\section{Magic Divine}
Divine magic, as the name implies, stems from the Gods.
The divine power of the gods may be sent across great distances, loaned to their servants so they may perform great works.
It may also be used directly, to smite an evildoer or destroy an abomination, although this happens only rarely.
Divine magic is the safest and most consistent magic for mortals to use, carrying little risk of destruction.
Obtaining favor from a god, however, may be difficult, and clerics must primarily live the life of a servant.

Divine magic is seen by many wizards as a more 
\end{multicols}
 



\part{The World's Locations}
\chapter{The Core}
At the center of Spirals is The Core.
This ancient structure is where the root tendrils meet.
It is the most magical place in Spirals, and also the most populated.

The Core takes the shape of a large, oblong spheroid, with a radius of roughly 716,000 miles.
Root tendrils protrude straight from the surface, spiraling away into space as they move outwards from it.

\section{The Great City}
The Core is covered in the works of mortals, active merchant districts existing next to ancient ruins.
Many people live on the core, but most simply travel through it, using it as a means to get from one tendril to another.


\section{The World Temple}
\label{worldtemple}
The Core is not a solid mass.
Beneath its surface lays a massive, holy structure: The World Temple.

The name is somewhat inaccurate.
Rather than being a temple to one god, it is a complex of temples to \textit{every} god.
Each deity from every pantheon has their own room, of varying size.
For the gods, this is their holiest place of worship, and a symbol of their divine power.

\subsection{Existent, not Created}
The temple was not built by hands mortal, or even divine, but by the universe itself.
It is a place so deeply magical it is impossible to comprehend or study.

The magic of the temple overpowers that of even the gods.
Although they may do as they please in their own temples, they cannot influence the temples of others.
Mortals, too, are bound by the rules of the temple, and those who gain enough power to challenge the gods cannot do so inside the temple.

\subsubsection{Severe Punishments}
Those who violate the rules of the temple are not smote, but punished.
These punishments are horrible, irreversible, and poetic in nature---and, although they would not admit it, the gods themselves fear them.

Perhaps the most notable of these punishments was inflicted on the human paladin Dygromac and his followers, some ten thousand years ago.
He was a deeply holy man, a friend to all, and a protector of the innocent.
So great was his justice that he possessed power to rival the gods, and his followers were said to be almost angelic in nature.

When the followers of the god Bane tortured and mutilated his love, Dygromac became filled with holy rage.
He lead his army into the temple, to destroy the god once and for all.
The righteous army made it within Bane's temple, Dygromac's thirst for justice so powerful as to shield them from the god's wrath.
For an instant, it looked as if the paladin would be triumphant---but, the instant he drew his sacred blade, he collapsed to the ground.
His features, and those of his followers, changed, taking on the colored skin, horns, and tails of the demonic.
All of Dygromac's power was stripped from him, and he became but a mortal once more, completely at the mercy of the god of Tyranny.

Bane, for his part, let the former holy man (and his follows) go free, knowing that a life of disgrace would be worse than any punishment he could inflict.
Dygromac's holy army was no more.
In their stead was a new race, the Tieflings, doomed to forever share the shameful origin of their forefathers.

\subsubsection{Neutral Grounds}
Certain portions of the World Temple, such as the hallways connecting individual temples, belong to nobody.
In these rooms, all forms of violence or magical trickery are prohibited.
They act as a ground for negotiation, taunting, and a wide variety of other interaction. 

\subsection{A true home}
Each god's temple is perfectly tailored to their personality.
The temple of Lolth, for example, is dark and damp, with spiderwebs dressing the halls and poisoned daggers hanging from the ceiling.
That of Silvanus is teeming with life, bright light filtering through thick vines, without an inch of stone visible.

The gods feel most at home in their temples, although their divine business often forces them to travel.
Even so, whenever they can, they dwell within their rooms, either in solitude, or surrounded by worshippers.

\subsubsection{An annoyance of layout}
The one downside for the gods is how their rooms are organized.
Whether out of a desire for balance or some twisted sense of humor, the temple organizes its rooms such that the gods that hate each other most are closest.
The entrance to Tiamat's temple is a mere ten feet away from Bahamuts, and Bane and Mysteria's temples share a wall.

This occasionally leads to chance encounters between hated enemies.
Bound by the rules of the temple, these encounters most often result in nothing more than an uncomfortable lack of acknowledgement, although two rivals will occasionally become involved in a verbal battle of wits.
In one case, Corellon and Lolth made known their distaste of each other via verse, reciting spontaneously-invented sonnets for several days.



\chapter{Major Tendrils and Formations}
Mapping out all the tendrils of spirals is a pointless task.
They flow onwards into infinity, multiplying seemingly infinitely and never ending.
It can be useful, however, to give a brief rundown of the most important tendrils

\begin{multicols}{2}




\section{The Great Flow}
The Great Flow is the name given by the citizens of Spirals to the largest water tendril, and the only root tendril created entirely out of water.
One of the six root tendrils, the Great Flow starts at a giant portal and flows outwards from the center of Spirals, branching only occasionally.
The tendril is around twenty miles wide and a perfect cylinder.
Unlike most of the other root tendrils, it does not travel in a straight line, opting instead to spiral and twist through the air as it travels.

The saltwater of the tendril hosts a variety of marine life, ranging from small fish near the edges to gargantuan sea monsters in the center.
Some sentient races fish the waters of the tendril, although the strong current makes this endeavor rather difficult.

\section{Frozen Tendril}
The aptly-named Frozen Tendril is the coldest of all the root tendrils.
From the second it leaves the center, the tube-shaped tendril is coated in ice and snow.
It would be completely abandoned, were it not for the great quantities of gold that appear in veins underneath its surface.
As a result of this potential wealth, both the outer and inner surfaces of the tendril are home to dwarven mines---and the occasional white dragon.

This tendril's light cycle takes much longer than normal, with nights lasting up to 1440 hours during the colder months, with extremely short days.
Even when the light is present, the tendril stays freezing, barely increasing in temperature.

\section{The Caldus Tendril}
The Caldus tendril is a vast desert landmass, and one of the six root tendrils.
It is a flat tendril with only one navigable surface.
The tendril is home to massive sandstorms, and is generally inhospitable to life.
Even so, some believe that the tendril may be home to a lost civilization or some kind of ancient treasure.

The nights on Caldus last about an hour, with days as long as thirty hours.
Nights, as a result, provide very little shelter from the blistering heat.

The tendril would be completely abandoned were it not the most efficient way to get to Syngland.
Even though the planetoid is barely a hundred miles out from the Core, anybody daring to make the trip for trade purposes will no doubt need to stop at least once to quench their thirst and rest.
The most unusual city of Sethi provides this.

\subsection{Syngland}
Syngland is a planetoid located along the Caldus tendril, about five miles across.
It is well known for its Silk production, as well as for being one of the very few Tiefling settlements anywhere.
Unlike the Tendril it floats off, it is quite humid in climate, and several degrees cooler.
Most of the planetoid is a large silk farm, where magical silkworms spin the finest threads in all of Spirals.

Of course, since it's mainly populated by Tieflings, and a rumors of its true purpose abound.
Many think that the silk is actually provided by some demon source, and stories of small children being kidnapped, taken to Syngland, and forced to spin until they die are common in Nursery books.

\subsection{Sethi}
Sethi is one of the extremely few settlements on the Caldus tendril.
It is located about halfway between the core and the Syngland Planetoid.
The city is used almost exclusively by merchants delivering goods from Syngland, with the only permanent residents being a clan of Dragonborn.
This clan was blessed by a goddess long ago, who gave them a special portal that led within The Great Flow.
The high priest of the clan is able to control the rate of this at will, although the goddess specified that it should not be open too quickly.
The clan uses this water to provide an oasis along the harsh desert---for those with enough gold, of course. 



\section{The Kapatagan Tendril}
The Kapatagan Tendril, one of the six root tendrils, is a rectangle around thirty miles across on each side.
The corners of the tendril are all coated in water, forming four streams that flow outwards from the origin point.
The easy availability of water, combined with the mild climate, cause the tendril to be coated with plants and ripe for farming.
The typical settlers are human closer to the core, and halfling as one more farther away. 

At some points, the grass fields turn into thick forests.
Most attempt to get through those as fast as possible, to avoid death by the creatures that live within.

\subsection{Kheti}
Kheti is a major settlement on Kaptagan, focused mainly on farming.
Initially created by monks wishing to provide the residents of the Sacred Core with food, it quickly grew in size until it was a major farming community.

Kheti provides the entire city of The Core with food, ranging from grains to livestock.
The once peaceful city, however, grew a dark underbelly as it grew in size.
Illegal horse races take place in shady tracks, where criminal loan sharks are all too willing to give the desperately addicted money.
Some claim that ranch buildings are home not to cows, but slaves or drug factories.

\subsection{Crash}
Crash is an offshoot tendril from the Kapatagan tendril, about three hundred miles from its origin at the core.
The tendril is approximately one hundred miles long, and takes the form of a wide, thin rectangle, nine miles across on one pair of sides, and a quarter mile on the other.
Although once greatly populated, the crops were quickly overtaken by thick, thorny vines.
The settlers left, believing it cursed, and there no longer exists any major settlements on its surface.
Sentient creatures rarely venture inwards.

\section{The Dzan Tendril}
The Dzan Tendril is a wet, humid Tendril covered in jungle.
It has short days and nights, with each stage lasting around six hours.

The tendril is moderately populated.
Its inhabitants include Elves, Dwarves, and Lizard-folk.
Settlements do not tend to last very long, however.
Typically, a city may survive for only a few decades before the residents decide to move to a more fertile spot.
This leaves the jungle full of ruins, causing the Dzan tendril to be a popular location for adventure stories.
\section{The Pacca Tendril}
The pacca Tendril is a cold, rainy tendril which meets the core near the Kpatagan tendril.
It is shaped like a large letter U, and curves around itself extremely slowly.
The large amount of rainfall cause its surface to be full of lakes and rivers, many of which provide enough fish to support a town or city.

While not freezing, the tendril's temperature are low enough that the people tend to wear heavy clothing.
Most inhabitants are from the heartier races as well, and elves are a rare sight on the Tendril.
The Tendril is best known for its bards, who follow the example set by the Tendril's most famous resident, David of the Burn, who used his music to defeat both a Lich and a dragon over the course of his life.

\section{The Missing Tendril}
 The orientation of the root tendrils suggests that there should be one more, between the Kapatagan and Caldus Tendrils.
No such Tendril, however, has ever been known to exist.

Wizards have long found trace magical energies where this Tendril would have met the core, leading many to believe that there used to be a Tendril in this location.
This Tendril is commonly referred to as "The Missing Tendril" or "The Seventh," and is the subject of conspiracies, ghost stories, and religious tales the world over.

\end{multicols}

\chapter{Minor Formations}


\section{The Confused Mountain}
The confused mountain is one of the stranger structures in spirals.
It is formed by six tendrils, three which branch off the Frozen Tendril, and three which branch off the Caldus.
They spiral around each other, forming the mesh outline of a cone and meeting at the center.
The result is a mountain-like structure that's half frozen and half desert, with stripes of either terrain running up its side.
The Confused Mountain is a popular subject for artists due to its natural beauty and inherent commentary on the nature of life, and is a symbol of the struggle between fanaticism and indifference in several religions.

\part{The World's Inhabitants (and their Works)}
\chapter{Gods}
The gods of Spirals are extremely close to the activities of mortals.
They are often physically present in the \hyperref[worldtemple]{World Temple (described in \autoref{worldtemple})}, and most have numerous clerics.
The pantheon of Spirals takes inspiration from that of Forgotten Realms, although some familiar faces have different personalities.


\begin{multicols}{2}

\section{Lawful Good Gods}
\subsection*{Bahamut}
\begin{goddesc}
\Level{Major}
\Domains{Knowledge}
\Symbol{The head of a dragon, in silver}
\index{Gods!Bahamut}
\end{goddesc}
Bahamut, the lord of good dragons, is a powerful force for good in the world of Spirals.
An ancient and powerful God, he has been alive for as long as mortals have recorded history, and still interacts with the world on a daily basis.
Unlike his Sister, he cares for all creatures, great and small, and fights daily so they may live their lives in a world filled with justice and peace.

\subsubsection*{Personality}
Bahamut is a kind soul with a firm personality.
It is rare to see a smile grace his features, and he can be stern in his dealing with his clerics.
This is not out of any distaste for mortals---indeed, he cares for the lives of those beneath him more than most of his draconic worshippers, and is far more active in their affairs than he is obligated to be.
Instead, his emotionless composure is borne of the grave nature of his task: evil is everywhere, and fighting it is a constant, endless struggle.

In certain forms, the dragon's personality may be altered.
When traveling in the disguise of an old man, for example, he can seem almost mad, laughing for no discernible reason and sprouting vague prophecies.
Some see this is a sign of inner turmoil, while others believe it to merely be a form of stress-relief.

\subsubsection*{Physical Appearance}
Bahamut normally takes the form of a colossally large dragon, majestic and beautiful in the same way as a piece of great architecture.
His scales shimmer and glow as if light by moonlight, and his dark grey eyes seem to contain storms.

Bahamut occasionally travels in other forms as well.
Most notably, he often mingles among mortals in the form of a human, appearing as either an insane old man wearing tattered robes or a handsome noble in a simple cloak.


\subsubsection*{World Temple Room}
Bahamut's complex in the world temple is impressive, so much so that many clerics who do not worship the platinum dragon make stops to visit it when on official business for their own guards.
The marble walls are coated in artwork, and every room has at least one sculpture.
A grand red carpet leads to the throne room of the god himself, where he sits surrounded by yet more art, as well as a carefully-organized collection of history books.
A door behind his throne leads to the room where he keeps his own horde, which consists largely of enchanted weapons, along with coins and other treasure.

Bahamut's temple is not only inhabited by the god himself, but also his most trusted servants.
A variety of ancient clerics live in the temple, performing divine research, surveying the outside world, and occasionally even answering prayers on their master's behalf.
These clerics have their own rooms, as well as common working areas that they all share.

The vast majority of those invited to live with the god are dragons, but Bahamut will occasionally grace a member lesser race this honor if they perform great feats in his name.
For a cleric loyal to the platinum god, this is an honor unimaginable.

\subsection*{Byr'nke the Wise}
\index{Gods!Byr'nke}
\begin{goddesc}
\Level{Intermediate}
\Domains{Knowledge}
\Symbol{Three gold coins surrounding an eye}
\end{goddesc}
God of money and monetary policy, Byr'nke the wise is a respected, if little-known, deity.
Whenever wise and prudent treasurers discuss their currency, his presence is felt, guiding their gentle hands.

\subsubsection*{Personality}
Byr'nke is a kind, if somewhat distant, god.
His clerics help in just financial institutions the world over, guiding their decisions and easing their extreme stress.
Some even rumor that Byr'nke's avatar will occasionally instruction students on the subject of economics at several colleges throughout the world.

\subsubsection*{Physical Appearance}
Byr'nke takes the form of a humanoid in banker's robes.
His exact species changes depending on the audience: elven when among elves, human amidst humans, and dwarven in the presence of dwares.
Some have even claimed that he has draconic or demonic forms, although they are rarely used if they exist.
The only common feature among his forms is his facial hair, a short goatee.

\subsubsection*{World Temple Room}
Byr'nke's room is extremely small.
it consists of several offices, occupied by his clerics, and one head office, occupied by himself.
There is also a boardroom where he meets with financial leaders or other gods.
The table in this room was donated to him as a gift from Bahamut, and is quite impressively made of solid gold.

\subsection*{Tecumtaz}
\index{Gods!Tecumtaz}
\begin{goddesc}
\Level{Major}
\Domains{War}
\Symbol{A flaming hammer}
\end{goddesc}
Tecumtaz is the god of just war, and perhaps one of the strangest gods in Spirals.
Wherever mortal fight tyranny and oppression, he is there, but he is not a god of kindness, and is barely one of restraint.

\subsubsection*{Personality}
Tecumtaz hates war.
This simple fact burns throughout his personality and actions.
Unlike other deities of war, who respect great warriors and bless mighty conquerors, the mightiest soldier in Tecumtaz's eyes is the one who ends the conflict the most quickly and justly, honor be dammed.
This hatred seeps into all other parts of his personality.
The face he presents to those who righteously end conflicts is one of supreme kindness, almost familial in nature.
Regardless of their status, he respects those people above all others.

To those fools who start senseless conflict, however, Tecumtaz presents a very different face, one full of terror and fury beyond all mortal comprehension.
This includes many other gods, even ones who supposedly share his alignment and domain.

If a cleric of Tecumtaz is to join a war, one can expect absolute chaos---scorched-earth tactics, destruction of civilian property, and large-scale destruction, although they will firmly prevent the deaths of the innocent.
One will also expect that, for better or for worse, the war will soon be over.

\subsubsection*{Physical Appearance} 
Tecumtaz most often takes the form of a man in an officer's uniform.
When enraged, his eyes glow with terrible fire.

\subsubsection*{World Temple Room}
Tecumtaz's room in the world temple takes the form of a military encampment at dusk, a vast field filled with several tents.
The air inside is cool and damp.
Tecumtaz himself takes position within the officer's tent, which is only slightly bigger than the rest.
Any clerics who happen to be residing with him stay in other tents.

\subsection{Chaotic Good Gods}
\subsection*{Corellon}
\begin{goddesc}
\Level{Major}
\Domains{Protection, War}
\Symbol{An elven ear}
\end{goddesc}
Corellon Leviathan, god(dess) of good elves, is known to almost all in Spirals.

\subsubsection*{Personality}
Corellon is a kind, just ruler.
They show mercy to the innocent, and work tirelessly to stop their former servant, Lolth.

\subsubsection*{Physical Appearance}
 Corellon is somewhat unique among gods in spirals in that their gender changes, seemingly randomly, every few decades.
 They find this somewhat stressful, as do their followers.
 Most elves refer to them as the gender that they grow up with, which can actually be used to determine their ages.
 
 \subsubsection*{World Temple Room}
 Corellon's world temple room is a forested area, full of trees which sprout from the marble floor.
 The complex is extremely large, and he almost always has a few clerics working within.
 
 Unlike many gods, Corellon choses to keep his throne extremely close to the entrance of his room.
The living chair sits some sixty feet away from the entrance, flanked on both sides by oak tress in the shape of elk.

\subsection*{Tyonir}
\begin{goddesc}
\Level{Major}
\Domains{Protection}
\Symbol{The silhouette of an upturned face}
\end{goddesc}
Tyonir is somewhat interesting among the gods.
Once he was the avatar of war looting, a hated and feared entity who helped hordes as they consumed and pillaged the lands they overran.
Over time, however, doubt grew in his mind, and soon he was having secret meetings with other deities with decidedly more just domains. 
In a crucial moment, Tyonir made a decision: none would suffer unjustly because of him ever again.
Now he continually struggles towards that goal, seeking to atone for his past sins while also acting as the god of redemption.

\subsubsection*{Personality}
Tyonir, even in his past form, was a decidedly more intelligent deity than most orcish gods.
He justified looting using an intellectual argument based on survival of the fittest and worthiness to possess goods, and armies under his guidance were horrifically effective at destroying and pillaging.
This is not to say that he was a cold, emotionless robot---far from it, the passion and fury his clerics felt in battle ran almost horrifically deep, enabling them to commit acts of depravity so unspeakable that they were rarely recorded in detail.

Tyonir's transformation was as much a change of mind as it was one of heart.
He slowly realized that his intellectual justifications for his actions were, under scrutiny, flimsy at best, while also gaining a new sense of empathy for his victims.

Tyonir is uncomfortable with passion, or, indeed, emotion in general.
It simply reminds him too much of his past actions in the fury of battle.
As a result, the god tries to be distant and intellectual, leaving the thirst for justice up to his clerics and paladins.

One emotion, however, consistently digs through his cold exterior: his own immense guilt.
After seeing the destruction of an orc army that used tactics he devised, or being exposed to some object that reminds him of a specific sin he committed in the past, Tyonir is likely to spend long periods alone.

Most of Tyonir's devout followers walk a similar path to the god.
Previously reprehensible people, they swear oaths of justice, and continually struggle to bring good to the world to atone for their past actions.
Often also wracked with guilt, they are more prone to sacrifice than most, and may struggle with their own urges to regress to their previous patterns of behavior.
Those that do regress and break their oaths almost always receive a rather short, painful visit from the god himself.

\subsubsection*{Physical Appearance}
Tyonir appears as an orcish man, about seven feet in height.
He wears the sackcloth of mourners, as a sign of his regret.
Around his face, however, he wraps a scarf of white linen, given to him as a joint gift from many of the other good gods---a symbol of his hope for the future.
He rarely takes this covering off, only showing his true face to those victimized by the views he once espoused.

\subsubsection*{World Temple Room}
Tyonir's complex in the world temple takes the form of a brightly lit, circular room, around a hundred yards in diameter.
Clerics sleep on simple cots, as does the god himself.
Around the room are pools of pure water which remove curses from artifacts and cleanse the mind, open for all who are pure of heart (or who have sworn to struggle to become so) to use.


\section{Lawful Neutral Gods}
\subsection*{Kurtulmac the Forgotten}
\index{Gods!Kurtulmac}
\begin{goddesc}
\Level{Minor}
\Domains{None applicable}
\Symbol{A kobold's head}
\end{goddesc}
Kurutlmac, the god of Kobolds, is quite possibly the least powerful deity in all of Spirals.
Much like the race he watches over, he is timid, meek, and ultimately worthless.

\subsubsection*{Personality}
Despite his near-total lack of power, Kurtulmac is a surprisingly personable god.
He is fully aware that he is useless, and uses self-deprecating humor to take advantage of this fact.
He is also extremely kind to his Kobolds, especially after their deaths (which happen quite frequently).

\subsubsection*{Physical Appearance}
Kurtulmac looks like a glowing Kobold, slightly taller than the normal members of his race.
He has red scales, and speaks with a cockney accent.

\subsubsection*{World Temple Room}
Much like the god himself, his room in the world temple is meek and small.
Approximately sixty square feet, it consists of little more than a throne made of bronze and a few torches.

\subsection*{Tediorus}
\index{Gods!Tediorus}
\begin{goddesc}
\Level{Minor}
\Domains{Order}
\Symbol{A shopping list}
\end{goddesc}
Tediorus is the god of menial tasks.
Taking out the garbage, sweeping a floor, exterminating rats---all such things belong to this god's domain.

\subsubsection*{Personality}
Tediorus is quite possibly the most boring thing in all of existence.
They feel no emotion or passion, instead executing the duties of godhood with almost robot-like efficiency.

Tediorus has some clerics, who seek to increase the amount of menial work in the world due to their belief that it "builds character."
This leads to them intentionally setting up menial tasks to be done: filling basements with giant rats, conjuring garbage to be put away, and disorganizing filing cabinets.
Adventurers who are just starting out on their journey may visit one of his temples to train their fundamental skills and to prepare themselves for the more difficult tasks of the profession.

\subsubsection*{Physical Appearance}
Tediorus, as one might expect, looks average.
He is exactly average height, has light-brown hair and brown eyes, and no distinctive facial features whatsoever.

\subsubsection*{World Temple Room}
Tediorus's World Temple room is little more than a white cube.
Tediorus sits in the middle on a plain-looking rocking chair, not speaking as he sews patches into shirts or mends handheld tools.

\section{Chaotic Neutral Gods}
\subsection*{Avarianice}
\begin{goddesc}
\Level{Major}
\Domains{Trickery}
\Symbol{A pile of gold coins}
\end{goddesc}
Avarianice is the goddess of greed and wealth accumulation.
She is worshipped both by dragons and lesser races for the graces she gives, however, her help always comes at great cost.

\subsubsection*{Personality}
Avarianice, as her job implies, is supremely greedy.
She is loath to part with even the smallest part of her glorious wealth, and will go to great lengths to punish those who steal even a single copper coin from her coffers.
She cares little for pride or honor, and is not beyond committing horifics acts to gain more wealth.
She is not, however, necessarily completely vindictive.
She is not beyond theft but does not use it as her primary source of income, and even her most deep-seated grudges can be quickly alleviated if the price is right.

Most clerics of Avarianice are slaves, paying off their debts by serving the goddess.
They serve more as collectors and bounty hunters than they do as protectors, shaking down others who owe debts and employing a variety of coercive tactics to receive payment.
Most are highly unhappy with their jobs.

Some clerics of Avarianice are true fanatics instead of slaves, crazed men and women who truly believe that their goddess deserves all their is in this world.
These clerics are her most powerful, and her most dangerous.
To them she grants her ultimate ``blessing": \textit{Hordesickness}.
Hordesickness is an obsession with gaining wealth that rivals that of the goddess.
This obsession comes with a shrewd ability to read people, fury in battle, and a keen sense of the value of objects, at the expense of terrible consequences for failing to obtain wealth..

Avarianice is said by some to be the root cause of the greed of dragons.
Supposedly, she cursed the entire race with a minor form of Hordesickness long ago, which eats away at them to this very day.
Bahamut and Tiamat both deny this.

\begin{adventureidea}{Hordesickness}
Avarianice's blessing is normally reserved for her most trusted clerics, however, she may sometimes ``grace" others with her power, in some circumstances.
If a player obtains hordesickness, they gain advantage on insight checks relating to gaining wealth, an immediate and accurate knowledge of the value of almost any item, and an extra $ 1d4 $ damage on melee attacks (if the attack has the potential to lead to income).
As a downside, however, they \textit{have} to gain wealth.
If they fail to increase their net worth by 5\% each weak, they lose $1d8$ maximum health.
This required increase compounds, and missed weeks \textit{do} count to the total required wealth.
If one's maximum health is reduced to zero by this effect, they die, and their soul is transferred immediately into Avarianice's possession.

This is an \textbf{extremely severe curse which will severely limit the decisions available to the person cursed.}
We recommend that you inflict it only on a highly temporary basis.
\end{adventureidea}

\subsubsection*{Physical Appearance}
Avarianice appears as either an extremely large dragon or an attractive young woman, depending on her audience.
In all cases, she is draped with the finest lace, and adorned with the rarest gems on all parts of her body.

\subsubsection*{World Temple Room}
Avarianice's world temple room is a great vault, hundreds of feet wide and long.
She sits there, on a golden throne atop a mountain of coins, very rarely seeing visitors.



\section{Lawful Evil Gods}
\subsection*{Tiamat}
\begin{goddesc}
\Level{Major}
\Domains{Knowledge}
\Symbol{Five Dragon Heads in Profile}
\end{goddesc}

Tiamat is the dragon goddess of greed and wealth, as well as the mother of all dragons.
You most likely recognize her---she's present in many D\&D settings, although her personality in Spirals is  altered.
Tiamat, like Bahamut, is an extremely old God, and has spent most of her long life becoming impossibly, ludicrously wealthy.
Tiamat sees the acquisition of wealth as a sacred duty, which dragons have a responsibility to exercise.

\subsubsection*{Personality}
Tiamat is a pragmatic deity, perhaps more so than any other draconic diety, or even most dragons.
She looks at her wealth over the long term, and is more than willing to part with small parts of her horde to increase its size later on.
It is not uncommon for her to loan her clerics rare magical items, scrolls, or other useful items.
She has even been known to give those exceptionally loyal to her small gifts, to incentivize similar behavior in others.

Tiamat views her pride as another component of her wealth, but her pride is not fragile.
The insults of a small creature far away are almost certainly not worth the time to incinerate them, and she respects the few who have more power than her.

\subsubsection*{Physical appearance}
Tiamat typically takes the form of a dragon with five heads, one for each chromatic color.
She occasionally also takes the form of a beautiful young woman, dressed in stupendously valuable clothing.

\subsubsection*{World Temple Room}
Tiamat's room in the world temple takes the form of an elegant palace.
The walls are lined with Platinum, carved into intricate murals, and decorated with fine art.

Behind her throne, in a room only she may access, Tiamat keeps the bulk of her horde.
Few have set eyes on it, and fewer have spoken of it afterwards.
What can be said for certain is that it is staggeringly large, and contains items the likes of which can hardly be described.

\subsection*{Bane}
\begin{goddesc}
\Level{Major}
\Domains{Trickery}
\Symbol{A cloud of smoke}
\end{goddesc}
Bane, the god of fear and Tyranny, is one of the most hated (and most worshipped) gods in the Spirals pantheon.

\subsubsection*{Personality}
Bane's personality is surprisingly charismatic.
He is confident, oddly funny, and even cordial.
Even those who hate what he stands for must admit that he is quite the charmer.
Tyranny, of course, is best executed by a leader who is at least somewhat liked, and fear is most effective when one's guard is down.

\subsubsection*{Physical Appearance}
Bane normally takes the shape of a handsome man, in his mid-twenties, wearing a black robe.
When he wishes to personally spread fear, he takes on an amorphous form, like black smoke.
This form quickly morphs to depict what the viewer fears most.
When in this form, all that view him must succeed on a DC 30 wisdom saving throw, or become frightened for 1d8 days.

\subsubsection*{World Temple Room}
Bane's world temple complex is extremely sparse.
It consists mainly of his throne and a few portraits of himself.
All else is located behind a small door, which has his symbol on it.
Many have been brought behind the door---none returned in any state to tell the outside world what they saw.

\section{Chaotic Evil Gods}
\subsection*{Dysnolar}
\begin{goddesc}
\Level{Major}
\Domains{War}
\Symbol{A rusted-out helmet}
\end{goddesc}
Dysnolar is the god of fighting relentlessly.
Embodying the unending stamina some soldiers seem to have for killing, he is a terrifying monstrosity with very few clerics.

\subsubsection*{Personality}
Dysnolar seems to have no ability to speak, or to think.
All he does is fight, manifesting on the battlefield during the most intense battles.
There is no recorded history of him speaking, although some witnesses have reported that he has said the single word "why" in a wet, raspy voice, like that of a man with a punctured lung.

\subsubsection*{Physical Appearance}
According to legend, Dysnolar was once a handsome warlord god, well-known for his good looks and charm, even as he was committing terrible acts.
All traces of this form, if it ever existed, are now completely gone, supposedly the result of a fight with Tecumtaz that Dysnolar ultimately lost.

Dysnolar currently appears as a huge suit of armor, thirty feet tall or more.
The armor is rusted and covered in blood stains, and it wields a shattered longsword.
Inside the armor there is nothing but black smoke, which swirls and twists rapidly as if inside a tornado.
Some say that this smoke is actually made of the souls of his clerics, who join their horrible patron after their deaths.

\subsubsection*{World Temple Room}
The appearance of Dysnolar's world temple room is unknown.
Nobody has ever entered and left alive.
From the doorway, all that can be seen is a black void, with the faint sounds of screaming coming from within.

\end{multicols}
\chapter{Items and Spells}
\todo[inline]{This chapter needs a lot of work}
\begin{quote}
\itshape
Kelvin Scyfront, wizard extraordinaire, swears in Elvish as he urges his Landcycle onwards.
Trees and bushes blur into a solid streak of green as his machine flies down the road.
\end{quote}

\section{Vehicles}
\subsection{Landcycles}
The primary form of personal transportation in Spirals is the landcycle.
These vehicles use the magic that holds the spirals up to travel long distances at extreme speeds.
Landcycles are a common piece of technology, and can generally be acquired cheaply.
They can be made from various materials, ranging from wood to metal to strange matter.
However, they have a few important catches:
\begin{description}
\item[Limited Carrying Capacity:] Landcycles can carry only approximately four hundred pounds.
If one wishes to transport a larger weight, they need to look for other options.
\end{description}


\todo[inline]{Landcycles in-game are much, much faster than this table. Fix it!}
\begin{table}[hb]
\caption{Types of Landcycle}
\begin{center}
\begin{tabular}{|c | c | c | c | p{0.07\linewidth} | p{0.1\linewidth} | p{0.1\linewidth}| }
\hline
Name & Material & Cost (GP) & Range (mi) & Speed (MPH) & Downtime (hours) & Startup Time (minutes) \\
\hline
Jaunter & Wood & 100 & 200 & 45 &10 & 15 \\
\hline
Cityrunner & Wood/Metal & 200 & 500 & 60 & 7 & 15 \\
\hline
Trekker & Wood & 600 & 800 & 60 & 10 & 20 \\
\hline
Harlot & Iron & 1,000 & 1,000 & 75 & 15 & 10 \\
\hline
Blackbird & Steel & 10,000 & 2500 & 250 & 10 & 5 \\
\hline

\end{tabular}
\end{center}
\end{table}

\subsection{Landchewers}
A slightly less common (and much more expensive) form of transportation is the Landchewer.
While not as fast, or as maneuverable, they are able to transport much heavier goods.

\todo[inline]{This table also doesn't match reality}
\begin{table}[hb]
\caption{Types of Landchewer}
\begin{center}
\begin{tabular}{|c|c|c|c|c|p{0.1\linewidth}|p{0.1\linewidth}|}
\hline
Name & Material & Cost (GP) & Range (mi) & Speed (MPH) & Downtime (Hours) & Max Weight (tons) \\
\hline
Tortoise & Iron & 10,000 & 200 & 45 & 24 & 0.75 \\
\hline
Ripper & Iron & 20,000 & 250 & 60 & 24 & 2.0 \\
\hline
Hauler & Lead/Iron & 35,000 & 400 & 70 & 40 & 5 \\
\hline
Gnasher & Lead/Iron & 100,000 & 900 & 100 & 40 & 20 \\
\hline
\end{tabular}
\end{center}

\begin{multicols*}{2}
\section{Spells}

\begin{spell}{Improved Illusory Script}
\Type{Second-Level Illusion}
\CastTime{10 minutes}
\Range{NA}
\Components{M (paper and ink)}
\Duration{Instantaneous} \\
When you first learn this spell, you generate two strings of magical runes: a \textit{private key} and a \textit{public key}.
These keys are magically linked.
Generally you will keep the private key in your mind, and write the public key down on a piece of paper.

You can give a copy of your public key to other creatures who know this spell, and receive their keys from them as well.
Once you have somebody else's key, you can send messages to them across any distance, including on other planes.
It takes approximately ten minutes to prepare a message for transmission, which you do by transcribing it onto a separate piece of parchment. 
Once transcribed, you the paper disappears in a puff of flame, and appears in front of the creature you wished to transmit to.
The transcribed paper can only be read by creatures who know the private key associated with the public key you transmitted to.
It takes approximately two minutes for creatures who know this key to decipher the message.
Before it is deciphered, it appears as a string of random letters, numbers, and punctuation.
\textit{Dispel Magic} does not remove this effect, nor does any source of true sight.

Creatures may tell others their private key, although this is highly dangerous, and should almost never be done.
Your private key can not be deciphered by mind-reading spells, and being told to reveal the key counts as damage for purposes of charm spells.

\end{spell}

\section{Misc. Items}

\subsubsection*{Elvesbane}
Elvesbane is a feared, legendary plant.
Said to be made from the essence of Lolth herself, it grows in the darkest, deepest places of the underdark.
When one soaks the root of an Elvesbane plant in boiling water, the resulting brew is a brutally efficient poison which works exclusively on elves.
With just one sip, an elf is doomed to die as all of their internal organs liquify---and no magic can save them.
Legend has it that drinking an entire pot would kill even Correllon him/herself.

\end{multicols*}


\end{table}

\chapter{Culture}
The world of Spirals, like all other worlds, has its own unique culture.
This will probably influence your character creation.

This chapter contains both general and specific cultural descriptions.
The more specific descriptions are provided to assist you in designing your own unique subcultures.

\begin{multicols}{2}
\section{Religion}
The various faiths and cults in the world of Spirals are, like everything else, effected by its geography.
Various tendrils are considered holy to different deities, depending on their composition.

\subsection{The Core}
The most major difference comes from the temple within the Core. 
The temple allows clerics and worshippers to have a much more direct location with their gods.
If one is particularly troubled, they may make a pilgrimage to the temple, to ask their god what to do.
It is unlikely, of course, they that will get to meet the god themselves, but praying near their physical presence tends to be more effective---and most of the world temples are filled with helpful clerics.

\section{Magic Itself}
Owing partially to the dangerous nature of casting spells, many wizards employ enchantments as opposed to directly casting spells.
The huge amounts of ambient magic present throughout Spirals allows simple spells which require only power to function to thrive.
Mechanically simple motion or heat charms have the most practical uses, and are by far the most commonplace.
Most notably, they are used in Landcycles and Landcrawlers, which you can read about in the technology section of this document.

Spells which require higher levels of precision are much harder to contain within an object.
If one wished to create a sword which cut through enemies more efficiently, for example, they would need to take care to temper the enchantment so it draws only the required amount of energy from the surrounding area.
Failure, of course, would result in the sword exploding violently the moment the enchantment was cast.
As a result, enchanted weapons and items are rarer than in other settings, as they are more difficult to make.

It should be noted, however, that the high levels of ambient magic increases the overall potential for enchanted items.
Few, if any, wizards have ever lived who were skilled enough to take advantage of this potential, but, if they could, the results could be wondrous (or devastating) indeed.\footnote{Yes, this does mean that you can give your favorite player a "+10 STR +30 DEX Sword of Epic Awesomeness" without breaking setting lore, although it may have a negative effect on your game.}

\section{Magic Users}
As outlined in the \hyperref[chapter:magic]{chapter on magic}, Magic behaves differently in Spirals than in many other worlds.
This, naturally, comes with some different cultural attitudes towards those who use it.

\subsection{Wizards}
Wizarding in the World of Spirals has a unique place, owing to the unique role of magic in the world.
Magic in Spirals is more prominent in Spirals than it is in other worlds, but this prominence adds danger.
The risk inherit in spellcasting makes wizarding a very mentally taxing job, both because of the risk and the mental fortitude required to overcome it.
Wizards can be good-natured, but the stress inherit in their profession makes many irritable. 

In order to reduce this risk and manage the stress of using magic, wizards in Spirals are carefully trained.
The process of becoming a wizard is extremely difficult, requiring nearly a decade of schooling and mentorship.
Wizards jealously guard major spells, and can be extremely competitive in their knowledge of the arcane, but they have sympathy for each other as well.
Wizards will generally be friendly with each other, assuming they have no major differences in worldview or goals, and will provide help to each other in times of need.

In their old age, wizards look forward towards their legacies, and backwards at their origins.
Older wizards who are not consumed with research will often seek to train apprentices or students, and many of the most powerful wizards will eventually teach at the College De Arcana.

\subsubsection{Interactions with Society}
Wizards are generally respected, if slightly feared.
For most citizens in Spirals, they will see very few wizards in their lives, although they may use tools crafted by them on a daily basis.
Wizards are viewed as learned beings, but the risks involved in using magic lead many to view them as having a certain undercurrent of imprudence or recklessness.
Many would eat dinner with them in a public place, but few would invite them into their home.

\subsection{Warlocks}

Warlocks gain their power not from careful study, but from a patron.
They rely on both their own knowledge of magic, and their patrons, to cast spells, although the patron has significantly more control.
Their patrons chose them well, and they are generally safe from the stress of arcane spellcasting---although their patron may cause them stress of a different variety. 

\subsubsection{Interactions with Wizards}
Many wizards view the arrangement between a Warlock and his Patron as a kind of cheating.
Wizards are distrustful of Warlocks, and generally avoid them.

Some of the more ambitious wizards are actively performing research into Warlock patronage, with the end goal of exploiting the methods patrons use to control magic for their own gain.
Patrons typically distrust these wizards.

\subsection{Sorcerers}
Sorcerers derive their magic from their own nature.
For reasons unknown, magic is especially active around them, pressing in on their minds, visibly moving through their bodies.

Sorcerers are more commonplace in Spirals than in most other worlds.
Unfortunately, however, they rarely survive until adulthood.
For most, the magic is too much for them, and it consumes them early in childhood.
Those who survive until adulthood are reasonably stable, however, it is not unheard of for them to lost control and burst into flames as well.
The author of this handbook does not recommend that you have this happen to any of your player characters.
\footnote{There are exceptions to this, for example, rolling 10 or more critfails consecutively, or when you desperately need to get rid of a player with a particularly foul personality or odor.}



\subsubsection{Interactions with Wizards}
Wizards generally view sorcerers with pity, mixed with suspicion.
While they won't say it in polite company, Wizards often refer to them as``Powderkegs", ``Flares", or ``Kindling," although this is done more as sympathetic gallows humor than anything else.

Long ago, wizards learned that the magic of sorcery could not be controlled via conventional wizarding means: the sorcerers who tried all died in spectacularly violent ways.
Since then, many wizards have sought to study sorcery, in the hopes of one day understanding its causes.
The end goal varies from researcher to researcher: some wish to harness its power, some wish simply to understand, and others wish find some way to magically reduce the mortality rate.
As of yet, however, sorcery remains a mystery.


\end{multicols}

\end{document}