\chapter{The Core}
At the center of Spirals is The Core.
This ancient structure is where the root tendrils meet.
It is the most magical place in Spirals, and also the most populated.

The Core takes the shape of a large, oblong spheroid, with a radius of roughly 716,000 miles.
Root tendrils protrude straight from the surface, spiraling away into space as they move outwards from it.

\begin{multicols}{2}
\section{The Great City}
The Core is covered in the works of mortals, active merchant districts existing next to ancient ruins.
Many people live on the core, but most simply travel through it, using it as a means to get from one tendril to another.


\section{The World Temple}
\label{worldtemple}
The Core is not a solid mass.
Beneath its surface lays a massive, holy structure: The World Temple.

The name is somewhat inaccurate.
Rather than being a temple to one god, it is a complex of temples to \textit{every} god.
Each deity from every pantheon has their own room, of varying size.
For the gods, this is their holiest place of worship, and a symbol of their divine power.

\subsection{Existent, not Created}
The temple was not built by hands mortal, or even divine, but by the universe itself.
It is a place so deeply magical it is impossible to comprehend or study.

The magic of the temple overpowers that of even the gods.
Although they may do as they please in their own temples, they cannot influence the temples of others.
Mortals, too, are bound by the rules of the temple, and those who gain enough power to challenge the gods cannot do so inside the temple.

\subsubsection{Severe Punishments}
Those who violate the rules of the temple are not smote, but punished.
These punishments are horrible, irreversible, and poetic in nature---and, although they would not admit it, the gods themselves fear them.

Perhaps the most notable of these punishments was inflicted on the human paladin Dygromac and his followers, some ten thousand years ago.
He was a deeply holy man, a friend to all, and a protector of the innocent.
So great was his justice that he possessed power to rival the gods, and his followers were said to be almost angelic in nature.

When the followers of the god Bane tortured and mutilated his love, Dygromac became filled with holy rage.
He lead his army into the temple, to destroy the god once and for all.
The righteous army made it within Bane's temple, Dygromac's thirst for justice so powerful as to shield them from the god's wrath.
For an instant, it looked as if the paladin would be triumphant---but, the instant he drew his sacred blade, he collapsed to the ground.
His features, and those of his followers, changed, taking on the colored skin, horns, and tails of the demonic.
All of Dygromac's power was stripped from him, and he became but a mortal once more, completely at the mercy of the god of Tyranny.

Bane, for his part, let the former holy man (and his follows) go free, knowing that a life of disgrace would be worse than any punishment he could inflict.
Dygromac's holy army was no more.
In their stead was a new race, the Tieflings, doomed to forever share the shameful origin of their forefathers.

\subsubsection{Neutral Grounds}
Certain portions of the World Temple, such as the hallways connecting individual temples, belong to nobody.
In these rooms, all forms of violence or magical trickery are prohibited.
They act as a ground for negotiation, taunting, and a wide variety of other interaction. 

\subsection{A true home}
Each god's temple is perfectly tailored to their personality.
The temple of Lolth, for example, is dark and damp, with spiderwebs dressing the halls and poisoned daggers hanging from the ceiling.
That of Silvanus is teeming with life, bright light filtering through thick vines, without an inch of stone visible.

The gods feel most at home in their temples, although their divine business often forces them to travel.
Even so, whenever they can, they dwell within their rooms, either in solitude, or surrounded by worshippers.

\subsubsection{An annoyance of layout}
The one downside for the gods is how their rooms are organized.
Whether out of a desire for balance or some twisted sense of humor, the temple organizes its rooms such that the gods that hate each other most are closest.
The entrance to Tiamat's temple is a mere ten feet away from Bahamuts, and Bane and Mysteria's temples share a wall.

This occasionally leads to chance encounters between hated enemies.
Bound by the rules of the temple, these encounters most often result in nothing more than an uncomfortable lack of acknowledgement, although two rivals will occasionally become involved in a verbal battle of wits.
In one case, Corellon and Lolth made known their distaste of each other via verse, reciting spontaneously-invented sonnets for several days.

\end{multicols}
