\chapter{Tendrils}

\begin{quote}
\itshape
Alysha the Warlock takes the moment to rest, laying down in the grass and looking up.
Above her, land swirls through the sky, like passageways in a great ant colony.
Over the course of her short life, she has walked, ridden, and fled upon many.
Yet, now, with their vastness above her, she feels just as she did back in Ryngolang, a young elf helping her mother do laundry and dreaming of the world beyond.
\end{quote}

Spirals is perhaps most unique for how its land is organized.
Spirals does not take place on a planet, or even several planets.
In fact, planets in the universe are a rarity.
Instead, matter is organized into twisting, irregular shapes, moving throughout reality in an infinite variety of different ways.
These shapes are commonly referred to as \textit{Tendrils}.

The tendrils are not a completely unorganized mess.
There are nine main tendrils, the arteries of the world.
They start at a central location, called \textit{The Core}, and grow outwards, growing in width as they travel apart.

Those tendrils eventually branch off into many more, very shortly after leaving the core.
These branches themselves have branches, creating a fractal pattern of land that criss-crosses the sky like a giant spider web.

This is not to say that the tendrils are a perfect mesh network.
Some are not attached to anything, floating alone in space.
Some twist back on themselves, or spin off in random fractal directions.
Some may even take the form of floating spheres, cubes, or other solid polygons.

Tendrils are a fairly unique concept.
In this chapter, we discuss how they work, how to make your own, and other topics you need to run your game.

\begin{multicols*}{2}
\section{Physics}
The Tendrils that make up the land (and sea) in Spirals are great masses of material floating in a magical aether.
They are made up of many materials, ranging from solid rock to loose dirt to water.
Some tendrils are even made entirely of air or other gasses, although those are obviously harder to detect.

The tendrils have gravity, but it works a bit differently than it does in real life.
The gravity of a tendril is primarily a function of the magic keeping it up.
As a result, some tendrils can be tubes as little as ten feet around, but still have gravity to hold large objects or creatures safely on their surface.
Tendrils which are close to each other may have gravitational fields so small that one can hop from one to the other, naturally finding their feet pulled to the floor as they transition.

Tendrils have a wide range of sizes. 
Some of the smaller ones can be thin pillars of earth that a typical human could wrap their hand around, while others can be so thick they take years to encircle on foot.

\subsection*{Lighting}
Tendrils experience a day and night cycle, just like in reality.
Unlike in reality, however, the light generally does not have a defined source.
Lighting and heat are a function of the same magic that holds up the tendrils, and patches of day move across them in strips several thousand miles wide.
Each strip takes around sixteen hours to move across a tendril during the warmer months, with a "night" lasting around six hours.
During the colder months, the night can last up to twelve hours, with day taking as few as five.
Different tendrils may have different timings.

\subsection*{Tendril Collapse}
The tendrils of Spirals are, in the general case, incredibly stable.
Physical forces seem to have no effect on the actual position of the landmasses, and even the most powerful of explosions will, at most, leave a large crater or obliterate a chunk of a thinner spiral, leaving both ends intact.
Under rare circumstances, however, a tendril can collapse completely.
Typical causes for this are:
\begin{description}
\item[Intentional Destruction:] groups of extremely powerful wizards may be able to collapse a tendril entirely by unwinding the magical framework in which it sits. 
This is occasionally done by governments to non-inhabited tendrils, in order to make trade routes easier, or (occasionally) to gain the raw materials that make up the Tendril.
Doing this costs an extremely large amount of money, as it requires the equivalent of several dozen \textit{wish} spells.
\item[Extreme Magical Disturbance:] certain magical acts may cause the magic holding a tendril up to unravel on its own, destroying a Tendril.
This happens so incredibly rarely that it has only happened in four cases that were well-documented.
\item[Acts of the gods:] the more powerful gods may chose to collapse a tendril to punish mortals. It should be noted, however, that collapsing a tendril with people on it is one of the most extreme actions a god can take. 
Even chaotic evil gods may think twice before doing so---especially because doing so is a great way to unite their enemies against them.
\item[Natural Decay:] some tendrils simply lose power over time, and will collapse on their own. This process, however, takes quite a long time, and those who are remotely magically sensitive will easily be able to detect what is happening a year or more before it does.
That is not to say that natural collapse never causes issues---a year is a rather short time period to move a city, much less an entire empire.
\end{description}

\begin{adventureidea}{Tendril Collapse}
Tendril collapse can provide an interesting plot hook for your own adventures.
Perhaps a crazed villain decides to destroy a tendril for his own material gain, or simply because he does not like it.
Perhaps the players must convince a group of wizards to help them collapse a spiral that houses an evil dragon or tarrasque, before it wakes up.
Perhaps a spiral has suddenly collapsed, killing thousands and leaving the survivors pointing fingers at their rival groups, looking for somebody to blame.
\end{adventureidea}


\section{Creating a Tendril}
The world of Spirals is home to uncountable tendrils, making their way across the stars.
Some of the major ones are described in this document.
However, the situation may arise in your campaigns where you wish to throw your players onto a brand-new tendril, that you created, or when you want a unique Tendril for a character backstory.
Here, we will provide some general guidelines for doing so, as well as roll tables to help you if you get suck (or if you just like the element of randomness).

\subsubsection*{Few Limitations}
The nature of tendrils gives you extreme freedom as a DM.
In fact, a tendril is able to have essentially any shape, size, environment, or other characteristic that you desire.

As such, the limitations mainly exist in terms of \textit{narrative consistency}.
A tendril should not logically break the society of Spirals.
For example, a tendril made of solid gold that's thousands of miles long would reduce the value of gold to near-worthlessness, due to the laws of supply and demand.
Putting such a tendril in your campaign may not be a good idea---unless you can justify it.
For example, perhaps the surface of the tendril is completely impenetrable, mocking those who attempt to mine it.
Maybe it is extremely far away from any settled area, accessible only via a long and near-suicidal journey through a tendril that alternates between bitter ice and scorching desert, and your players are simply the first to reach it.
Perhaps everybody knows it exists, but it is floating by itself far from any other landmass, through a vacuum that is unaccessible for even flying creatures, with an aura around it that prevents magical teleportation.\footnote{Such a tendril is almost guaranteed to be irresistible to your players, likely eventually resulting in a spaceship mission to acquire its wealth. Of course, a space race campaign seems to have a lot of potential, so perhaps that's a good thing.}

In truth, almost nothing is completely off-limits.
If you have an idea for an especially unique Tendril, just be sure to think through its logical consequences.
You may even want to run your idea past others, to get a second opinion on the matter.


\subsection{Location}
Tendrils branch off from one another, creating a tree with infinitely many branches.
The root tendrils have the most offshoots, although each offshoot can have its own offshoots.
Tendrils may also split into multiple, simply end, or appear disconnected from any others.

A tendril will typically originate from another Tendril that is similar in climate, although this is not a hard rule---occasionally, a searing desert with transition directly into a frigid tundra.

\begin{rolltable}{Tendril Locations (d12)}
\item[1-3] Offshoot of the root tendril most similar in climate, near the core
\item[4-5] Offshoot of the root tendril most similar in climate, far downtendril
\item[6]  Offshoot of the root tendril most different in climate, near the core
\item[7] Offshoot of the root tendril most different in climate, far downtendril
\item[8-10] Offshoot of a random root tendril 
\item[11] Offshoot of a minor tendril
\item[12] Floating alone in space, accessible only by a portal
\end{rolltable}


\subsection{Shape}
Tendrils come in many shapes.
Depending on the story you want to tell, these shapes may be vitally important to the plot, or mostly incidental.
If you simply want a jungle adventure, for example, a jungle Tendril of any shape will probably suffice.
However, this is not always the case.
Tendrils of different shapes can be used to affect gameplay and story significantly.

Let's think of a Tendril which is a rectangular prism.
People on one side of the tendril will have severe difficulty seeing other people (or creatures) that are on the other sides of the tendril, even if those people may be dangerously close to them.
Your players (or antagonists) may be able to use this to their advantage, in order to hide from a dangerous monster or to set up an ambush.


\begin{rolltable}{Tendril Shapes (d12)}
\item[1-2] Cylinder
\item[3-4] Rectangular Prism
\item[5-6] Hollow cylinder, with walkable inside surfaces on both sides
\item[7-8] Half pipe
\item [9-10] Flat surface
\item[11] Planetoid
\item[12] Floating chunks of flat land
\end{rolltable}

\subsection{Climate}
Tendrils vary widely in climate, from dense jungle to arid desert.
Some are even composed entirely of water.

Tendrils tend to be fairly consistent in climate across their surface, although some do change.
Tendrils that suddenly undergo an extreme transition (such as from tundra to desert)  may be considered to actually be two tendrils pressed together.

\begin{rolltable}[0.5\textheight]{Tendril Climates (d10)}
\item[1] Underwater
\item[2] Jungle
\item[3] Desert
\item[4] Forest
\item[5]  Swamp
\item[6] Tundra
\item[7] Plains
\item[8] Ice
\item[9] Rocky
\item[10] Volcanic
\end{rolltable}

\subsection{Complications}
Tendrils are landmasses, but they are magical landmasses.
As such, they can occasionally have certain \textit{unique} features.

These features can be mundane or highly exciting, useful or hindering, logical or unusual.
We recommend that you give Tendrils complications sparingly, so that the ones which do have them are special.

\begin{rolltable}{Tendril Complication Type (d6)}
\item[1] Negative Complication
\item[2-5] No complication
\item[6] Positive complication
\end{rolltable}

\todo{Finish this roll table}
\begin{rolltable}{Negative Tendril Complications (d12)}

\item[1-2] The tendril is highly infested with monsters at a much greater rate than normal.
\item[3-4] The tendril is cursed, and nothing edible to sentient races can grow on it.
\item[5-6] The tendril suffers from extreme and frequent Earthquakes.
Every minute, roll a d20.
On a 1, an Earthquake happens, and all creatures on the ground must make a DC 15 DEX save to keep their footing.
\item[7-8] The tendril has a high amount of volcanic activity, and lava oozes up from cracks frequently.
\item[9-10] \textbf{Need this}
\item[11] \textbf{And this}
\item[12] Roll twice on this table, and use both complications.
\end{rolltable}



\todo{Finish this as well}
\begin{rolltable}{Positive Tendril Complications (d12)}
\item[1-2] The tendril is extremely peaceful, and rest comes easily.
Only half of the usual time is required to complete long and short rests.
\item[3-4] The tendril is bountiful.
All checks to forage automatically succeed.
\item[5-6] This tendril is blessed by a good god.
Pick a god at random.
Any checks related to something within this god's domain have advantage.

\item[12] Roll twice on this table, and use both complications.

\end{rolltable}

\end{multicols*}