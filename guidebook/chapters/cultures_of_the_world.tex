% !TEX encoding = UTF-8 Unicode
\chapter{Cultures of the World}
\FancyLetter{T}he cultures of various races in Spirals are sometimes what you may expect from other worlds, with some minor twists
In other cases, they differ severely, sharing few traits besides name.

This chapter gives a rundown of some of the larger cultural groups in Spirals.
It is designed to be helpful both when creating characters, and when designing campaigns.


\begin{multicols}{2}
\section{Aarakocra}
\index{Aarakocra!Culture}
The graceful masters of the sky are an uncommon sight even in Spirals, staying secluded away from populated areas.
One traveling on Har will likely see small glimpses of them flying around, but they are careful to keep their cities inaccessible, and will only rarely make direct contact with the other races.

\subsection{Ancient and Secluded}
The Aarakocra are widely believed to be one of the oldest races in Spirals, created at least six billion years ago.
Unfortunately, their history is marked with tragedy and oppression.
The Aarakocra have been enslaved several times by various groups, their gift of flight making them highly desirable for back-breaking construction of great monuments.
Many of the dark temples and vile idols of the world were constructed with Aarakocra labor.
Although in the modern era the Aarakocra live free, they remember their time in bondage, and take steps so they never have to endure such conditions again.
Their cities are built in areas that are near-impossible to access without flying caravans, and heavily fortified.

\subsection{Golem-Builders}
The Aarakocra spend most of their time in the air, and are uncomfortable fighting on land.
As such, they have spent millennia honing their abilities in golem-crafting, creating artificial soldiers to do the job for them.
Aarakocra who demonstrates a talent for working with magic and clay are identified at a young age, and encouraged to join an apprenticeship with a master golem-crafter.
Such an apprenticeship is a great honor for an Aarakocra, and typically the cause of celebration for his or her entire extended family.

\subsection{An Ancient Religion}
The Aarakocra near-universally honor ``The One in the Sky", their name for Streeaire, god of flight.
It is this god that gives them their ability to fly, and who has guided them throughout the centuries.
The religion of the Aarakocra is centered around their great temple, located on a mountain-top in their capital city.
There they keep the Eternal Flame of Flight, a magical fire which acts as a connection to their god.
It is kept constantly lit with holy oil.
This oil is manufactured in the World Temple, and must be delivered to their capital by a consecrated, male priest of Streeaire.

Being a priest in the temple is a great honor, but also a great sacrifice.
The ceremonies required to keep the flame properly lit require a priest to burn some of his own pilot feathers, sacrificing his ability to fly so that people may continue to take to the air.
Only a select few are able to handle the mental stress of doing so, and, of that number, fewer are intellectually fit to follow the complex laws a priest must follow in order to keep himself spiritually pure.

\subsection{Special Circumstances}
In light of all this information, one may wonder why an Aarakocra would go adventuring.
Indeed, many Aarakocra ask themselves this question, viewing the members of their race who take up arms against evil as foolish, insane, or arrogant.
While it is true that an adventuring Aarakocra rarely happens, 
Still, an adventuring Aarakocra is not entirely unheard of.
The Aarakocra have a bit of a superstitious streak, and may occasionally come to view one of their community as ``cursed" or ``bad luck", which results in isolation.
Members who experience this are somewhat likely to leave their homes and seek their fortune among the other races, although they rarely wind up actively trying to get revenge on those who wronged them.

It is also not unheard of for an Aarakorca to go adventuring out of a sense of internal duty.
Aarakorca who are gifted with magical talents or fighting prowess may see it as their job to go out and better the world, so that their race will no longer have to live in fear.
These Aarakorca typically struggle with homesickness, especially when they are forced to walk or stay in confined areas for an extended period, but often pull through by force of will.

Aarakorca may also leave their homes for personal reasons.
Aarakorca can hunger for revenge, justice, or an answer just as much as the other races.

One of the most notable Aarakorca adventurers was Skretz Zial, the brother of an Aarakorca high priest.
Unable to handle the fact that his brother was no longer able to fly with him, he set off on a quest to determine some way to allow the Aarakorca to fly that did not require sacrifice.
Although this quest was ultimately unsuccessful, his example occasionally inspires others, who take up arms and explore the wild places of the world in his name.

\section{Dragonborn}
\index{Dragonborn!Culture}
Dragonborn in Spirals have one of the clearest origins as a race, being one of the newer races in Spirals.
Archeologists in Spirals have traced their genesis to three unique points in history: The Klaxax Army, the Marlin Mines, and the Longclaw/Rizzuto Crime Family.
The Klaxax army dragonborn appeared first, according to records, and the Marlin Mines dragonborn appeared some two hundred years later.
The Longclaw/Rizzuto dragonborn are by far the newest, having existed for potentially less than two thousand years.

\subsection{Origin: The Klaxax Army}
The now-defunct Klaxax Army is the oldest originator of dragonborn known to modern historians.
Most of the exact details are lost to history, but scholars know that, at one point in time, there was a large war between two opposing groups of metallic and chromatic dragons.
This war took place on the Dzan Tendril, fairly distant from the core, and lasted several hundred years.
These dragons initially fought alone, but both sides were soon joined by large human armies. 
Scholarly debate is contentious on whether or not these armies joined for ideological reasons, because of the promise of treasure, or simply political disputes of their own, but the human armies soon grew to such sizes that they eclipsed the fighting power of the dragons themselves.
The war then progressed to a state where the dragons handled strategy, and the humans did most of the fighting.

At some point, one of the armies began to intentionally create dragonborn for their in-built fighting abilities, and for how quickly they reached maturity.
Scholars are contentious as to how, exactly, this creation worked, although archeological evidence suggests that the number of dragonborn increased far too quickly for it to be any kind of natural means.
The other army quickly followed suit, and the end result was a rather large population of dragonborn, fighting on both sides.

The outcome of the war is lost to history, but the end result was two civilizations of dragonborn who had reached a seemingly permanent peace.
These civilizations initially stayed separate, but quickly mingled, and formed what is believed to be a joint Stratocracy of some sort.
This civilization stayed in power for several hundred years, but all their great cities are ruins now.

The dragonborn who once made up these civilizations, in the current day, are fractured across the Tendrils.
They do, however, share some common traits.
For one, they are slightly larger than most other dragonborn, and tend to be more muscular.
For another, their settlements typically follow a rigid social hierarchy, and many make their living as the base of operations for an army-for-hire.

\subsection{Origin: Marlin Mine}
The dragnborn of the Marlin Mine appeared after those of the Klaxax Army, and under decidedly less violent circumstances.
The Marlin Mine was built some ten thousand years ago to extract gold from a vein on the Har tendril.
This vein was legendarily pure and large, and is believed by most Dwarven scholars to have been the largest deposit of gold in its time.

A group of around two dozen metallic dragons initially discovered the source of gold, and quickly began ferrying human and dwarven miners to the area to extract it.
Is it unclear if the non-dragons were paid laborers or slaves, but, in either case, there was an extremely large number of them.

The first traces of dragonborn appear approximately twenty years into the mining operation.
By forty years, archeological records and recovered journals show that dragonborn made up a small but substantial portion of those working in the mines.
These dragonborn steadily increased in number until the vein was exhausted, upon which the mining colony was disbanded, and the dragonborn apparently set off on their own.
A substantial number appear to have returned with the laborers to their various homelands, although others set off on their own.

It is unclear why the dragons chose to create the dragonborn.
The most common idea is that the dragons needed the dragonborn to defend the mines from monsters, such as the Drow, who were rumored to lurk in the area.
Others think that they created the dragonborn to perform supervisory duties, acting as managers who were loyal to their draconic creators.
Malren Goldarm, a prominent dwarven scholar of the mining operation, had this to say, in a memorable outburst at a discussion on the topic held at the College D`arcana :

\begin{itquote}
Why'd they create the dragonborn? Have any of you gits ever even set foot in a mine like that, much less lived in one? This was before everyone and their dog had a bloody landcycle, they weren't gettin' the latest magazines delivered! What the hell else were they going to do to stay entertained?
\end{itquote}

Debate, unfortuantely, continues on this manner.

\subsection{Origin: The Longclaw/Rizzuto Crime Family}
The most recently created dragonborn come from the Longclaw/Rizzuto crime family.

The Longclaw clan was a criminal organization that operated on the core some twenty-five hundred years ago.
Although it had a variety of races under its employment, the top levels of the family were composed entirely of dragons.
The organization was involved in many crimes, including drug trafficking, arms dealing, extortion, racketeering, the slave trade, the smuggling of forbidden magical and religious artifacts, and money laundering.
They employed a complex scheme of bribing guards, threats, and, of course, murder to avoid justice.
Eventually, the Longclaw family grew so powerful that their \textit{Clanfather}, Tezoth, apparently had regular meetings with Tiamat herself, and over half of the businesses on the core were paying them some form of protection money.

The Longclaw clan avoided any competition in their criminal enterprises for over three hundred years, but eventually a challenger arose.
To the fury of the dragon leaders of the Longclaw, it was not another group of dragons, but instead a family of humans.
The Rizzuto got their start in the city of Kheti, committing many of the same crimes the Longclaw did, but with significantly different means.
The namesake Rizzuto family was highly magically gifted, with many of their children being born as sorcerers, and those who didn't often studying magic to become a wizard.
The Rizzutos were more flexible than the Longclaw, and allowed those who weren't related by blood to join into leadership positions as long as they married in.

The Rizzuto eventually grew bored with the small city, and soon moved to the Core to seek greater fortune.
They soon demonstrated brutal skill in evading both those who wanted to inflict justice upon them and the rival Longclaw clan.
What followed was a prolonged gang war which resulted in widespread death and property destruction.
Although the Longclaw had leadership that was more physically capable, they lacked the creativity (and brutality) of the newcomers, and soon found themselves loosing ground.
This, for obvious reasons, infuriated the Longclaw, who could hardly believe that they were loosing to humans.
The fighting only intensified, and soon large-scale battles were taking part on the street.

After twenty years of war, the families eventually realized that they were only hurting themselves, and peace talks began.
The leader of the Rizzuto, Donna Maria, eventually proposed a solution which would end the conflict: the two organizations would merge into one, which could rule the Core's underworld with an iron fist.
Clanfather Tezoth initially did not agree, but was essentially forced to after the Rizzuto destroyed several key warehouses full of illegal goods, leaving him unable to pay a debt towards somebody even he could not intimidate.\footnote{Who, exactly, was his creditor is unclear, although the most popular belief was that it was Tiamat herself.}

Donna Maria and Tezoth both agreed that there needed to be some connecting link between the two families besides a simple agreement, and soon arranged two marriages.
Tezoth's son, Briam, would marry Maria's granddaughter Bianca, and his daughter Parath would marry Maria's grandson Vincent.
These marriages were followed by others, as regional bosses tried to unify their respective workers.
Within a generation a surprising number of the leadership was half-dragon or dragonborn, and it spread out from there.

Nowadays, the crime family has gone underground, after a series of paladins worked to ``Clean up the Core".
Many of the dragonborn who were members of it spread out, throughout the land.
Some, however, are still in contact with their family, extorting merchants, murdering informers, and smuggling goods, just as they have for thousands of years.

\todo[inline]{Possibly add a background here, where you're part of this crime family. Prof. in Deception and Intimidation, some kind of feature where you are able to utilize the family to get contacts on the Core or something?}


\end{multicols}