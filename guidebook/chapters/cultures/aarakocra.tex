\index{Aarakocra!Culture}
The graceful masters of the sky are an uncommon sight even in Spirals, staying secluded away from populated areas.
One traveling on Har will likely see small glimpses of them flying around, but they are careful to keep their cities inaccessible, and will only rarely make direct contact with the other races.

\subsection{Ancient and Secluded}
The Aarakocra are widely believed to be one of the oldest races in Spirals, created at least six billion years ago.
Unfortunately, their history is marked with tragedy and oppression.
The Aarakocra have been enslaved several times by various groups, their gift of flight making them highly desirable for back-breaking construction of great monuments.
Many of the dark temples and vile idols of the world were constructed with Aarakocra labor.
Although in the modern era the Aarakocra live free, they remember their time in bondage, and take steps so they never have to endure such conditions again.
Their cities are built in areas that are near-impossible to access without flying caravans, and heavily fortified.
They are distrustful of the other races in general.

\subsection{Golem-Builders}
The Aarakocra spend most of their time in the air, and are uncomfortable fighting on land.
As such, they have spent millennia honing their abilities in golem-crafting, creating artificial soldiers to do the job for them.
Aarakocra who demonstrates a talent for working with magic and clay are identified at a young age, and encouraged to join an apprenticeship with a master golem-crafter.
Such an apprenticeship is a great honor for an Aarakocra, and typically the cause of celebration for his or her entire extended family.

\subsection{An Ancient Religion}
The Aarakocra near-universally honor ``The One in the Sky", their name for Streeaire, god of flight.
It is this god that gives them their ability to fly, and who has guided them throughout the centuries.
The religion of the Aarakocra is centered around their great temple, located on a mountain-top in their capital city.
There they keep the Eternal Flame of Flight, a magical fire which acts as a connection to their god.
It is kept constantly lit with holy oil.
This oil is manufactured in the World Temple, and must be delivered to their capital by a consecrated, male priest of Streeaire.
If it were to go out, all Aarakocra would lose the ability to fly until it is again lit, which would be utterly disastrous.

Being a priest in the temple is a great honor, but also a great sacrifice.
The ceremonies required to keep the flame properly lit require a priest to burn some of his own pilot feathers, sacrificing his ability to fly so that people may continue to take to the air.
Only a select few are able to handle the mental stress of doing so, and, of that number, fewer are intellectually fit to follow the complex laws a priest must follow in order to keep himself spiritually pure.

\subsection{The \textit{Ackyea}}
In order to maintain godly ``cleanliness", the Aarakocra are barred from any acts of violence that are not in response to an immediate threat. 
This is an issue, as it also prescribes the death penalty for murder and a variety of other severe crimes---not to mention the fact Aarakocra are quite fond of meat.

In order to get around this, Aarakocra villages designate a special member to be the \textit{Ackyea}, which roughly translates to "burdened."
This member is typically also made captain of the guard, and handles the slaughtering of animals, execution, and other, less pleasant tasks.
The \textit{Ackyea} is always a volunteer, receives special religious blessing and rites to protect their soul from their job.
The \textit{Ackyea} is still unclean, and is not allowed to maintain traditional religious services for the duration of their work.
Becoming an \textit{Ackyea} is a great sacrifice, and not one many are brave enough to undergo.

\subsection{Special Circumstances}
In light of all this information, one may wonder why an Aarakocra would go adventuring.
Indeed, many Aarakocra ask themselves this question, viewing the members of their race who take up arms against evil as foolish, insane, or arrogant.
While it is true that an adventuring Aarakocra rarely happens, 
Still, an adventuring Aarakocra is not entirely unheard of.
The Aarakocra have a bit of a superstitious streak, and may occasionally come to view one of their community as ``cursed" or ``bad luck", which results in isolation.
Members who experience this are somewhat likely to leave their homes and seek their fortune among the other races, although they rarely wind up actively trying to get revenge on those who wronged them.

It is also not unheard of for an Aarakorca to go adventuring out of a sense of internal duty.
Aarakorca who are gifted with magical talents or fighting prowess may see it as their job to go out and better the world, so that their race will no longer have to live in fear.
These Aarakorca typically struggle with homesickness, especially when they are forced to walk or stay in confined areas for an extended period, but often pull through by force of will.

Aarakorca may also leave their homes for personal reasons.
Aarakorca can hunger for revenge, justice, or an answer just as much as the other races.

One of the most notable Aarakorca adventurers was Skretz Zial, the brother of an Aarakorca high priest.
Unable to handle the fact that his brother was no longer able to fly with him, he set off on a quest to determine some way to allow the Aarakorca to fly that did not require sacrifice.
Although this quest was ultimately unsuccessful, his example occasionally inspires others, who take up arms and explore the wild places of the world in his name.