

Magic in Spirals, on the surface, is similar to magic in other worlds.
Mechanically it works much the same: it may be granted in a variety of ways, channeled into spells, and brewed into potions.
Beneath the surface, however, the way the world of Spirals interacts with its magic is quite different.

\begin{multicols}{2}
\subsection{Presence}
In Spirals, the presence of magic is felt in the lives of all creatures.
It is what holds the Tendrils together, gives them form and substance, lights them during the day, and, occasionally, creates them from nowhere.

\subsection{Arcane Magic}
When an arcane caster uses a spell in Spirals, they seek to move the ambient magic around them, to give it form and substance.
This is a dangerous task indeed---magic, at its core, \textit{desires} to be used, to make its mark on the world, and it eagerly rushes to a caster, attempting to make that desire reality.
Unfortunately for the caster, however, the eagerness of the magic to be used does not extend to a desire to be controlled, or to be limited.
Without careful focus and application of arcane techniques, the caster will lose control of her magic, and it will run wild within her.
This can have a wide variety of effects, but, unfortunately, the most common by far is to cause the caster to burst into flames.

\subsubsection{Enchanting}
Spirals has a lot of ambient magic, and clever wizards can tap into it, creating enchanted objects.
For simple enchantments this is not difficult.
If one wishes, for example, they could easily direct ambient magic to propel a vehicle or heat a bit of metal.
These simple energy transfer spells are reliable, and, as a result, vehicles like landcycles and magical forges are somewhat common.

Spells which require higher levels of precision, however, are much harder to contain within an object.
If one wished to create a sword which cut through enemies more efficiently, for example, they would need to take care to temper the enchantment so it draws only the required amount of energy from the surrounding area.
Failure, of course, would result in the sword exploding violently the moment the enchantment was cast.
As a result, enchanted weapons and items are rarer than in other settings, as they are more difficult to make.
It should be noted, however, that the high levels of ambient magic increases the overall potential for enchanted items.
Few, if any, wizards have ever lived who were skilled enough to take advantage of this potential, but, if they could, the results could be wondrous (or devastating) indeed.

\subsection{Magic Derived from Nature}
Those who walk in wild places may have magic of their own.
Unlike arcane magic, it is not derived from the listless sort of magic that simply exists, but from a more purposeful, directed form: that which gives nature its shape.
This magic is found in the ground of the Tendrils, and in the features on (or below) their surface.
It is more introspective, gentler than the arcane magic, granting wondrous abilities to those it deems worthy. 

\subsubsection{Magical Creatures}
The greater presence of magic results in a greater presence of magical creatures, as well.
Dragons in Spirals are still relatively rare, for example, but there are still enough of them that almost all the larger armies are able to contract a few for special operations work.
Horrifying monsters such as mind flayers or beholders, too, are in no short supply.
Giant animals, too, are fairly commonplace.

This seems like it may be cataclysmically disastrous for mortal civilization, but, in truth, it's not as bad as you'd think.
The magic that allows the world to be filled with wyverns and Slaads also brings up powerful heroes who can deal with such things.

\subsubsection{Nature without Life}
Unlike in other settings, the magic of nature is not necessarily the magic of life---Tendrils covered in parched desert or suffocating ice have magic in equal measure as those painted with plants and trodden upon by animals.
This magic, however, rarely allows itself to be used by those among the living, preferring the inanimate.
When it does allow itself to be used, it is generally out of self-preservation: the desire to have its domain kept as it likes it, with little life.
These servants rarely cause direct harm to the living, as the rotting of flesh is itself a haven for life, choosing instead to expel the living to more suitable places.
\label{druidofdecay}
Occasionally, however, a desolate place will grow discomforted, unsatisfied with its domain.
In these cases, a Druid of Decay may arise, blindly serving his or her arcane master, singly devoted to the goal of expanding desolation and removing life.
Most often these evils are quickly dispatched by those who preserve living nature, but they occasionally may have a larger effect.

\begin{adventureidea}{Desolation}
Druids of decay, when they arise, are a horrible plague upon the world.
They are often extremely powerful, and hell-bent on expanding the desolation of their home tendril into other, vibrant areas of life.
Worse, they often retain their sanity and ability to plan, making them dangerous foes.

A druid of decay can make a great antagonist for your campaign, especially because they can have such variance in personality.
Some are actually quite diplomatic: they'll give the populations of towns and cities a chance to retreat before destroying them, and might even intentionally scare off animals as they spread their desolation.
Others are significantly less charitable, and may intentionally cause as much death as they possibly can.
\end{adventureidea}

\subsection{Magic Divine}
Divine magic, as the name implies, stems from the Gods.
The divine power of the gods may be sent across great distances, loaned to their servants so they may perform great works.
The gods also use this magic directly when dealing with mortal affairs, although they only intervene in such a manner when absolutely necessary.

Divine magic is by far the safest type of magic.
Being completely regulated by a caster's patron, there is no risk that it will overcome the caster or spin wildly out of control.
Unfortunately, this safety comes with a caveat of lifestyle---one must serve a god or goddess in order to use it, and they near-universally require strict lives of service and works from their clerics.

Divine magic can also sometimes be used by those with extreme purity of will,\footnote{Note that, in this context, "pure" does not mean "holy" or "righteous", but "untainted." If one has an extreme and concentrated desire to commit horrible acts, they too can use such magic---often to horrifying ends.} such as paladins who have not sworn allegiance to any particular god.
This magic is equally safe as god-derived magic, but only as long as the user's will remains pure and focused on their task.

The nature of divine magic and arcane magic, on a fundamental level, seems remarkably similar.
Various energy tests performed at wizarding colleges have confirmed that the two varieties share almost all of their fundamental properties.
This has led some wizards to believe that gods are merely extremely magical beings.
Some have even attempted to ascend to godhood themselves via the use of ancient artifacts or complex rituals, a task which almost always results in nothing more than an extremely large explosion.
\end{multicols}
 
 
 \subsection{Variant Rule: Extremely Wild Magic}
The naturally wild nature of magic in Spirals makes being a sorcerer even more taxing.
Those who use wild magic may occasionally experience effects even more random than normal.

Using this variant rule, wild magic sorcerers get a very small chance to lose control of their magic in a \textit{ridiculous} way.
Whenever you roll a 1 on the wild magic dice, an additional D20, and a D8.\footnote{If you rarely have a wild magic sorcerer roll a wild magic dice, you may want to remove the d8 from the equation.}
If both are one, roll on the extremely wild magic table to see what sort of effect manifests.

Be warned that the effects of this rule will make your campaign quite silly.
Many will grant your sorcerer or one of his allies a new ability, permanently effect their appearance, or otherwise destroy any sense of narrative cohesion on your campaign.
The chances of getting to roll on the Supremely Wild Magic table are $ \frac{1}{3200} $, but the effects are potentially cataclysmic, and could derail your entire campaign in a glorious way.

\todo[inline]{Finish and edit this roll table}
\begin{rolltable}{Supremely Wild Magic}
\item[1-2] Roll on this table at the start of your turn for the next two minutes.
If you get this item again, the duration is refreshed.
Duration can be refreshed a maximum of twice.
If your roll lands on this, strap in.
You're about to see some serious shit.
\item[3-4] Gain a new class ability, \textit{Ultimate Perception.}
This ability may be cast after a short rest.
Roll a d20.
On a 20, you gain vision through the eyes of every creature within a 300-foot radius of where you cast it.
The effect lasts until the creatures die, or \textbf{you} leave the radius---the creatures may leave, and you retain vision.
\item[5-6] You burst into flames. You take four times your maximum health as damage.
You can chose one target within 10 feet of you to also take this damage.
If this spell kills you (which it most likely will), you instantly and inexplicably gain a cosmically significant position in whatever your afterlife would be, of your DM's choosing.
\item[6-7] A tarrasque appears 2d100 squares away from you. 
It heads for the nearest populated area at its maximum movement seed. 
The tarrasque disappears after two hours.
\item[7-8] You (and all allies within 100 feet) instantly change race in terms of appearance, gaining the new race's bonuses, and retaining those of your old race.
The DM may decide if you get to pick the new race, or if it is randomly selected.
\item[9-10] The world wants to kill you for 1d4 days.
Every five minutes, \textit{Meteor Storm} will be cast at whatever position you were at in the previous round, until you die, or the duration of the effect ends.
\item[11-12] On all subsequent level ups, the additional hit points you gain are doubled.
\item[13-14] Your appearance changes to that of a small archdemon.
You also gain the ability to cast \textit{Finger of Death} as a bonus action once a month.
\item[15-16] You gain \textit{hordelust.}
You will become physically sick if you do not increase your net worth by at least 5\% every week, gaining disadvantage on WIS and DEX saving throws if you fail to do so.
You gain advantage on all checks relating to the appraisal of wealth, as well as Persuasion checks when buying or selling items.
\item[17-18]  Roll a d4. Your height permanently increases by the result, in feet.
\item[19-20] The heat around you spontaneously vanishes.
All creatures in a 30-foot radius of you take 10d10 ice damage. 
If a creature dies as a result of this roll, their body becomes an ice sculpture.
\item[21-22] You feel the power of sacrifice within you.
You gain the ability to heal up to three creatures to full whenever you reduce yourself to negative hit points.
When you perform this sacrifice, you lose the ability to be magically healed until you are non-magically restored to consciousness.
Recharges on a long rest.
\item[23-24] You and all creatures you can see suddenly become magically attracted to a faraway tendril. 
You will fall towards it, upwards through the air, until you touch down on its surface.
Your falling speed is capped to 20 $\frac{m}{s}$.
The fall may take several days, or even weeks, at the discretion of your DM.
\item[25-26] You permanently grow feathers all over your body, and gain the ability to speak to birds.
\item[27-28] You permanently grow a pair of wings, gaining a flying speed equal to your walking speed if unarmored, or wearing armor specially deigned to allow you to use the wings.
You also gain a wing attack. 
It has a reach of 10 feet, has 3 plus your dexterity modifier to hit, and does 1d12 damage.
You may use this attack as a bonus action.
\item[29-30] All gold and platinum within 50 feet of you permanently vanishes.
You gain the ability to cause this effect as a standard action, with a recharge time of one week.
\item[31-32] All creatures you have witnessed the death of in the last week are instantly resorted to life as if by the \textit{reincarnation} spell.
\item[33-34] Creatures automatically fail all saving throws against your spells for the next 1d4 minutes.
\item[35-36] All plants and animals within a ten-mile radius are put under the effects of the \textit{Awaken} spell, minus the charm.
\item[37-38] Your physical form loses its cohesion.
You gain the ability to change you physical appearance to any creature you have seen within one size category of yourself as a bonus action, which also changes your voice.
When not imitating another creature, you appear blurry and ethereal.
\item[39-40] Your strength permanently increases by three, but you physically gain the appearance of somebody who is severely malnourished.
\item[41-42] You gain 8 eye-stalks, like a beholder. You get a +5 to wisdom checks that rely on sight.
\item[43-44] A new Tendril sprouts out of the ground 5 feet in front of you. The DM decides its properties.
\item[45-46] Magic screams into your veins, begging you to release it.
On your turn, you must expend half of your remaining spell slots on spells of your choosing, and cast all of your cantrips once.
Each spell is cast as a free action regardless of its normal cast time.
Immediately afterwards, you will fall unconscious for 4d8+12 hours.
You gain the ability to manifest this effect once every four months.
\item[47-48] You and all creatures within 400 feet become Tyrannosaurus Rexes.
You alone keep your ability to cast spells, although everybody keeps their normal stats and intelligence.
This effect may only be reversed by the usage of a \textit{Greater Restoration} spell, although it may be difficult to convince a cleric to cast it on your party---especially since you can't speak in anything that isn't a roar.
\item[49-50] The magic around you is so eager to help you that it forgets about gravity for a moment.
You (and all other creatures you can see) become weightless for fifteen minutes.
You may manifest this effect once a month.
\item[51-52] You gain the ability to speak in tongues, which allows creatures that understand any language to understand your speech.
You also gain the ability to understand all languages.
\item[53-54] All weapons in a 100-foot radius of you are transformed into foam replicas, which do 1d2 bludgeoning damage, but keep enchantments.
This effect manifests regularly in monthly intervals.
\item[55-56] From now on, critical hits on offensive spells you cast will create a 1d8 foot deep crater below the target.
\item[57-58] You permanently gain a dragonborn breath weapon, of your choosing. If you are already a dragonborn, you get another breath weapon.
\item[59-60] Every unoccupied space on the map is filled with a flumph, which is controlled by the DM and frightened of you.
\item[61-62] Time stops in a 400-foot radius (centered on you) for a duration of one year. Nobody may enter the zone of stopped time.
\item[63-64] Magic enjoys how you kill things.
Form now on, when you deal a killing blow, you regain health equal to the damage the blow did.
\item[65-66] A huge forest sprouts around you, along with an entire civilization of very confused Elves.
The Elves are aware that they were just brought into existence, but somehow have knowledge equivalent to their age.
\item[67-68] You gain a suit of huge magical armor for one minute. 
Your size is changed to huge for the duration, and your AC increases by +10.
You may manifest this effect once a month.
\item[69-70] You become immune to all mind-alerting substances.
\item[71-72] All your skin appears to fall off, revealing the muscles and organs beneath. 
Somehow, you are unharmed.
Your skin will grow back in 1d4 weeks, however, whenever you drop to zero hit points, it will fall off again.
Medicine checks against you have advantage while skinless.
You also gain advantage on intimidation checks, owing to your horrifying appearance, with the penalty of disadvantage on persuasion checks.
\item[73-74] You gain the ability to scream incredibly loudly.
This action causes creatures within 30 feet of you to take 2d8 thunder damage, and shatters any non-magical glass item in the radius.
It recharges on a short rest.
\item[75-76] You can trade your life force for time.
You gain the ability to become immune to all damage for one minute, instantly dropping to zero hit points at the end of the duration.
\item[77-78] All people of noble blood within 40 miles disappear.
The DM decides where they go, or if they die.
Those within the world temple are immune.
\item[79-80] You gain the ability to speak to small woodland creatures.
\item[81-82] A clone of you, with all your memories and abilities, appears right next to you.
It will die in one week, a fact which it knows, and is somewhat upset about.
It may have some requests of things it wants to do before it perishes, or it may seek to get revenge, depending on your personality.
\item[83-84] The spell gets lost on the way to its target.
Somehow, it manages to hit something very far away.
\item[85-86] HELP
\item[87-88] Everyone within 200 feet of you gain a permanent psychic link to each others mind and know what each of you are thinking at all times.
\item[89-90] HELP
\item[91-92] The next 1d8+1 times you die, you instantly come back to life, as if via the \textit{Reincarnate} spell. 
\item[93-94] HELP
\item[95-96] The world desires a comedy.
As a standard action, you can chose to create a 40-foot-square \textit{Zone of Bad Luck}, which gives disadvantage on \textbf{all} rolls, centered around yourself.
The zone of bad luck follows you around, and lasts for two minutes.
You may use this ability once a month.
\item[97-98] Your maximum sorcery points increases permanently by 1d12.
You also regain all expanded sorcery points.
\item[00-00] Everybody in your party gets 2d4+1 free levels in sorcerer, including you. These levels \textbf{do not count to the required XP to level up,} and are granted regardless of stat eligibility.
\end{rolltable}

