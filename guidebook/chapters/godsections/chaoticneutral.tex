\subsection*{Avarianice}
\index{Gods!Avarianice}
\begin{goddesc}
\Level{Major}
\Domains{Trickery}
\Symbol{A pile of gold coins}
\end{goddesc}
Avarianice is the goddess of greed and wealth accumulation.
She is worshipped both by dragons and lesser races for the graces she gives, however, her help always comes at great cost.

\subsubsection*{Personality}
Avarianice, as her job implies, is supremely greedy.
She is loath to part with even the smallest part of her glorious wealth, and will go to great lengths to punish those who steal even a single copper coin from her coffers.
She cares little for pride or honor, and is not beyond committing horifics acts to gain more wealth.
She is not, however, necessarily completely vindictive.
She is not beyond theft but does not use it as her primary source of income, and even her most deep-seated grudges can be quickly alleviated if the price is right.

Most clerics of Avarianice are slaves, paying off their debts by serving the goddess.
They serve more as collectors and bounty hunters than they do as protectors, shaking down others who owe debts and employing a variety of coercive tactics to receive payment.
Most are highly unhappy with their jobs.

Some clerics of Avarianice are true fanatics instead of slaves, crazed men and women who truly believe that their goddess deserves all their is in this world.
These clerics are her most powerful, and her most dangerous.
To them she grants her ultimate ``blessing": \textit{Hordesickness}.
Hordesickness is an obsession with gaining wealth that rivals that of the goddess.
This obsession comes with a shrewd ability to read people, fury in battle, and a keen sense of the value of objects, at the expense of terrible consequences for failing to obtain wealth..

Avarianice is said by some to be the root cause of the greed of dragons.
Supposedly, she cursed the entire race with a minor form of Hordesickness long ago, which eats away at them to this very day.
Bahamut and Tiamat both deny this.

\begin{adventureidea}{Hordesickness}
Avarianice's blessing is normally reserved for her most trusted clerics, however, she may sometimes ``grace" others with her power, in some circumstances.
If a player obtains hordesickness, they gain advantage on insight checks relating to gaining wealth, an immediate and accurate knowledge of the value of almost any item, and an extra $ 1d4 $ damage on melee attacks (if the attack has the potential to lead to income).
As a downside, however, they \textit{have} to gain wealth.
If they fail to increase their net worth by 5\% each weak, they lose $1d8$ maximum health.
This required increase compounds, and missed weeks \textit{do} count to the total required wealth.
If one's maximum health is reduced to zero by this effect, they die, and their soul is transferred immediately into Avarianice's possession.

This is an \textbf{extremely severe curse which will severely limit the decisions available to the person cursed.}
We recommend that you inflict it only on a highly temporary basis.
\end{adventureidea}

\subsubsection*{Physical Appearance}
Avarianice appears as either an extremely large dragon or an attractive young woman, depending on her audience.
In all cases, she is draped with the finest lace, and adorned with the rarest gems on all parts of her body.

\subsubsection*{World Temple Room}
Avarianice's world temple room is a great vault, hundreds of feet wide and long.
She sits there, on a golden throne atop a mountain of coins, very rarely seeing visitors.


