\subsection*{Kurtulmac the Forgotten}
\index{Gods!Kurtulmac}
\begin{goddesc}
\Level{Minor}
\Domains{None applicable}
\Symbol{A kobold's head}
\end{goddesc}
Kurutlmac, the god of Kobolds, is quite possibly the least powerful deity in all of Spirals.
Much like the race he watches over, he is timid, meek, and ultimately worthless.

\subsubsection*{Personality}
Despite his near-total lack of power, Kurtulmac is a surprisingly personable god.
He is fully aware that he is useless, and uses self-deprecating humor to take advantage of this fact.
He is also extremely kind to his Kobolds, especially after their deaths (which happen quite frequently).

\subsubsection*{Physical Appearance}
Kurtulmac looks like a glowing Kobold, slightly taller than the normal members of his race.
He has red scales, and speaks with a cockney accent.

\subsubsection*{World Temple Room}
Much like the god himself, his room in the world temple is meek and small.
Approximately sixty square feet, it consists of little more than a throne made of bronze and a few torches.

\subsection*{Tediorus}
\index{Gods!Tediorus}
\begin{goddesc}
\Level{Minor}
\Domains{Order}
\Symbol{A shopping list}
\end{goddesc}
\begin{itquote}
Somebody has to do it.
\end{itquote}
\QuoteAttr{Unofficial motto of Tediorus' Clerics}
Tediorus is the god of menial tasks.
Taking out the garbage, sweeping a floor, exterminating rats---all such things belong to this god's domain.

\subsubsection*{Personality}
Tediorus is quite possibly the most boring thing in all of existence.
They feel no emotion or passion, instead executing the duties of godhood with almost robot-like efficiency.

Tediorus has some clerics, who seek to increase the amount of menial work in the world due to their belief that it "builds character."
This leads to them intentionally setting up menial tasks to be done: filling basements with giant rats, conjuring garbage to be put away, and disorganizing filing cabinets.
Adventurers who are just starting out on their journey may visit one of his temples to train their fundamental skills and to prepare themselves for the more difficult tasks of the profession.

\subsubsection*{Physical Appearance}
Tediorus, as one might expect, looks average.
He is exactly average height, has light-brown hair and brown eyes, and no distinctive facial features whatsoever.

\subsubsection*{World Temple Room}
Tediorus's World Temple room is little more than a white cube.
Tediorus sits in the middle on a plain-looking rocking chair, not speaking as he sews patches into shirts or mends handheld tools.

\subsection*{Himl}
\index{Gods!Himl}
\begin{goddesc}
\Level{Intermediate}
\Domains{Tempest, Order}
\Symbol{A Golden Feather}
\end{goddesc}

Himl is the Aarakocran Lord of the Sky, and the 

\subsection*{Fonok}
\begin{goddesc}
\Level{Major}
\Domains{Trickery}
\Symbol{An empty chest}
\end{goddesc}

Fonok is the god of organized criminal activity.
Wether small scale gangs or large, spread-out mafia families, Fonok helps to guide all of those who follow codes to break laws.

\subsubsection{Personality}
Fonok is an honorable god.
He is unlikely to provide forgiveness, but will on occasion dispense mercy.
Fonok is fairly involved in the larger crime families, and acts as a mentor to those who lead them.

Fonok dislikes crimes of passion and revenge schemes.
Of course, those who disrespect the willing must pay for their actions, but Fonok would prefer that this is done cleanly and as ``ethically'' as possible.
In general, Fonok encourages his followers to only commit crimes that help the family or gang they are a part of.

Fonok is highly respectable and personable.
He enjoys opera, fine wine, and expensive clothing.
It's generally best to dress up as much as you can before seeking his help.

Fonok is okay with almost all crime, but has one steadfast rule: no kids.
This applies in pretty much any capacity, from hiring a street urchin to distract guards to much more horrendous crimes.
The quickest way to piss the god off is to involve somebody young in something big.

\subsubsection*{Physical Appearance}

Fonok appears as an older human gentlemen, almost always dressed in expensive robes or a fine suit.
He is slightly balding, and looks less attractive than most other gods.
He makes up for this, however, by being very intimidating.
Simply speaking forcefully without raising his voice is as terrifying as staring down a roaring dragon---especially since Fonok is even more likely to be able to back up his threats.

\subsubsection*{Responsibilities}
Fonok is responsible for keeping organized crime organized.
He provides advice and aide to those who have recently come to power in their organizations, and helps resolve disputes between different families.
Clerics of Fonok will help decide territorial boundaries, rules and regulations, and the other logistics behind organized crime.

Fonok is especially active in turf wars and other such conflicts.
He and his clerics seek to keep those conflicts as clean and professional as possible, least they descend into common criminality.

\subsubsection*{World Temple Room}
In honor of his followers, who are rarely able to rule their empires from a palace, Fonok's temple room typically takes the form of various hideouts.
Most often, it is a high-end restaurant, where he sits at the head of the largest table.
His private residence resembles a reasonably-sized mansion.
On personal business, he will take somebody aside to his room, but often has meetings at the table.

Fonok's world temple room is a place for crime families to resolve disputed diplomatically.
He maintains a strict ``no weapons'' policy for everybody but his personal clerics.

\subsection*{Gep}
\begin{goddesc}
\Level{Major}
\Domains{Knowledge}
\Symbol{A smokestack}
\end{goddesc}

Gep, occasionally titled ``The Iron King'' is the god of industry and manufacturing.
He is concerned with large-scale production of consumer and industrial goods.

\subsection*{Personality}
Gep is a slightly gruff, hard to read god.
Although he dresses well and occasionally hosts divine parties with expensive food and drink, he himself seems mostly unconcerned with these things.
Gep is a god of few words, and rarely discusses things outside of his domain.

Inside his domain, however, Gep is highly active.
He is knowledgeable about the latest production techniques, commodity prices, supply chain management, and everything else related to industry.
He's been known to go on long lectures about the merits of various shipping techniques, vertical vs. horizontal integration, price movements in raw materials, and ways to train laborers.

\subsubsection*{Physical Appearance}
Gep takes the form of a middle-aged dwarf.
He typically wears casual clothing in his day-to-day administration and only puts on the suits and robes for events with other gods or meetings.

\subsubsection*{Responsibilities}
Gep is responsible for managing industry and making it more productive.
To that end, his clerics act as practical advisors and consultants, going to factories and helping them run their business.
Gep is also interested in the science behind manufactoring, and helps those who are more concerned with theory to put their ideas to work.

Gep doesn't really care about what is being constructed.
He has no qualms about helping unsavory characters make illegal goods or goods to be used for an immoral purpose.

Gep dislikes those who are unproductive, but does not actively hate them.
In truth, he stops thinking about them the instant they leave the factory floor. 

\subsubsection*{World Temple Room}
Somewhat surprisingly, Gep's world temple room is not a factory floor.
To the contrary, he performed an analysis and realized that the additional shipping cost to get raw materials from the surface made his allotted slot essentially useless for any kind of manufacturing.
As a result, his room is an extremely barebones office, with several portals that lead to factories he or his clerics personally own.

