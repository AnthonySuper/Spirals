\subsection*{Corellon}
\index{Gods!Corellon}
\begin{goddesc}
\Level{Major}
\Domains{Protection, War}
\Symbol{An elven ear}
\end{goddesc}
Corellon Leviathan, god(dess) of good elves, is known to almost all in Spirals.

\subsubsection*{Personality}
Corellon is a kind, just ruler.
They show mercy to the innocent, and work tirelessly to stop their former servant, Lolth.

\subsubsection*{Physical Appearance}
 Corellon is somewhat unique among gods in spirals in that their gender changes, seemingly randomly, every few decades.
This happens for seemingly no reason---it is not a blessing, nor a curse, nor the result of any conscious decision on the part of the god.
They find this somewhat stressful, as do their followers.
Most elves refer to them as the gender that they grow up with, a fact which can can help approximate exactly when the a given elf was born.

Regardless of their gender at the time, Corellon is the embodiment of the grace and poise.
Slender and with nearly-ordered hair, they are often described as ``beautiful", and comparisons to the god(dess) are a staple of romantic poetry.

They have one other form, that of an extremely large Elk, some eight feet tall.
Confusingly, this elk form takes the \textit{opposite} gender of their elven form---a cow when Corellon is male, and a bull when they are female, which only serves to add further confusion.
 
 \subsubsection*{World Temple Room}
Corellon's world temple room is a forested area, full of trees which sprout from the marble floor.
The complex is extremely large, and he almost always has a few clerics working within.
 
Unlike many gods, Corellon choses to keep his throne extremely close to the entrance of his room.
The living chair sits some sixty feet away from the entrance, flanked on both sides by oak tress in the shape of elk.

\subsection*{Tyonir}
\index{Gods!Tyonir}
\begin{goddesc}
\Level{Major}
\Domains{Protection}
\Symbol{The silhouette of an upturned face}
\end{goddesc}
Tyonir is somewhat interesting among the gods.
Once he was the avatar of war looting, a hated and feared entity who helped hordes as they consumed and pillaged the lands they overran.
Over time, however, doubt grew in his mind, and soon he was having secret meetings with other deities with decidedly more just domains. 
In a crucial moment, Tyonir made a decision: none would suffer unjustly because of him ever again.
Now he continually struggles towards that goal, seeking to atone for his past sins while also acting as the god of redemption.

\subsubsection*{Personality}
Tyonir, even in his past form, was a decidedly more intelligent deity than most orcish gods.
He justified looting using an intellectual argument based on survival of the fittest and worthiness to possess goods, and armies under his guidance were horrifically effective at destroying and pillaging.
This is not to say that he was a cold, emotionless robot---far from it, the passion and fury his clerics felt in battle ran almost horrifically deep, enabling them to commit acts of depravity so unspeakable that they were rarely recorded in detail.

Tyonir's transformation was as much a change of mind as it was one of heart.
He slowly realized that his intellectual justifications for his actions were, under scrutiny, flimsy at best, while also gaining a new sense of empathy for his victims.

Tyonir is uncomfortable with passion, or, indeed, emotion in general.
It simply reminds him too much of his past actions in the fury of battle.
As a result, the god tries to be distant and intellectual, leaving the thirst for justice up to his clerics and paladins.

One emotion, however, consistently digs through his cold exterior: his own immense guilt.
After seeing the destruction of an orc army that used tactics he devised, or being exposed to some object that reminds him of a specific sin he committed in the past, Tyonir is likely to spend long periods alone.

Most of Tyonir's devout followers walk a similar path to the god.
Previously reprehensible people, they swear oaths of justice, and continually struggle to bring good to the world to atone for their past actions.
Often also wracked with guilt, they are more prone to sacrifice than most, and may struggle with their own urges to regress to their previous patterns of behavior.
Those that do regress and break their oaths almost always receive a rather short, painful visit from the god himself.

\subsubsection*{Physical Appearance}
Tyonir appears as a male orc, about eight feet in height.
He wears the sackcloth of mourners, as a sign of his regret.
Around his face, however, he wraps a scarf of white linen, given to him as a joint gift from many of the other good gods.
He rarely takes this covering off, only showing his true face to those victimized by the views he once espoused.

\subsubsection*{World Temple Room}
Tyonir's complex in the world temple takes the form of a brightly lit, circular room, around a hundred yards in diameter.
Clerics sleep on simple cots, as does the god himself.
Around the room are pools of pure water which remove curses from artifacts and cleanse the mind, open for all who are pure of heart (or who have sworn to struggle to become so) to use.


\subsection*{Eling}
\index{Gods!Eling}
\begin{goddesc}
\Level{Minor}
\Domains{Knowledge, Life}
\Symbol{An upturned trumpet}
\end{goddesc}

Eling is the god of musical improvisation and innovation.
Once a mortal musician, he was elevated to godhood at the time of his death.
The reasons for this are the subject of debate.
Some say the other gods saw an opportunity to fill a position and granted it to him, while others claim that the world simply couldn't bear the thought of him no longer playing.
Whatever the case, he has earned his place as a well-respected deity.

\subsubsection*{Personality}
Eling is a friendly, humble god.
He acknowledges that he is one the greater musicians of all time, but doesn't believe this is especially worthy of praise.
In his own words, ``there's only good and bad music---that's the only scale that matters, anyway.''

Eling specializes in highly syncopated, freeform melodies with heavy improvisation and interesting rhythms.
However, he listens to all kinds of music, and is enthusiastic about the subject in general.

\subsubsection*{Physical Appearance}
Eling appears as a young, dark-skinned man who is never seen without his trademark black mustache.
He is near-always dressed as a conductor.

\subsubsection*{Responsibilities}
Eling is responsible for the creation of new music, and especially new kinds of music.
His clerics are often musicians themselves, who prowl the world searching for new sounds.
Most of them are, of course, also bards.

Eling is perhaps most noted for occasionally acting as a composer and conductor of his own original material.
To play in his orchestra is one of the greatest honors a musician can receive---especially since the audience is often other deities.

\subsubsection*{World Temple Room}
Eling's world temple room is split in two parts.
The first is a set of small studios where musicians can safely create new sounds.
The second is a moderately-sized lounge, with a decently-sized stage.
Musicians will occasionally perform there for important audiences, and other gods sometimes visit in order to destress.


