\chapter{Items and Spells}
\todo[inline]{This chapter needs a lot of work}
\begin{quote}
\itshape
Kelvin Scyfront, wizard extraordinaire, swears in Elvish as he urges his Landcycle onwards.
Trees and bushes blur into a solid streak of green as his machine flies down the road.
\end{quote}

\section{Vehicles}
\subsection{Landcycles}
The primary form of personal transportation in Spirals is the landcycle.
These vehicles use the magic that holds the spirals up to travel long distances at extreme speeds.
Landcycles are a common piece of technology, and can generally be acquired cheaply.
They can be made from various materials, ranging from wood to metal to strange matter.
However, they have a few important catches:
\begin{description}
\item[Limited Carrying Capacity:] Landcycles can carry only approximately four hundred pounds.
If one wishes to transport a larger weight, they need to look for other options.
\end{description}


\todo[inline]{Landcycles in-game are much, much faster than this table. Fix it!}
\begin{table}[hb]
\caption{Types of Landcycle}
\begin{center}
\begin{tabular}{|c | c | c | c | p{0.07\linewidth} | p{0.1\linewidth} | p{0.1\linewidth}| }
\hline
Name & Material & Cost (GP) & Range (mi) & Speed (MPH) & Downtime (hours) & Startup Time (minutes) \\
\hline
Jaunter & Wood & 100 & 200 & 45 &10 & 15 \\
\hline
Cityrunner & Wood/Metal & 200 & 500 & 60 & 7 & 15 \\
\hline
Trekker & Wood & 600 & 800 & 60 & 10 & 20 \\
\hline
Harlot & Iron & 1,000 & 1,000 & 75 & 15 & 10 \\
\hline
Blackbird & Steel & 10,000 & 2500 & 250 & 10 & 5 \\
\hline

\end{tabular}
\end{center}
\end{table}

\subsection{Landchewers}
A slightly less common (and much more expensive) form of transportation is the Landchewer.
While not as fast, or as maneuverable, they are able to transport much heavier goods.

\todo[inline]{This table also doesn't match reality}
\begin{table}[hb]
\caption{Types of Landchewer}
\begin{center}
\begin{tabular}{|c|c|c|c|c|p{0.1\linewidth}|p{0.1\linewidth}|}
\hline
Name & Material & Cost (GP) & Range (mi) & Speed (MPH) & Downtime (Hours) & Max Weight (tons) \\
\hline
Tortoise & Iron & 10,000 & 200 & 45 & 24 & 0.75 \\
\hline
Ripper & Iron & 20,000 & 250 & 60 & 24 & 2.0 \\
\hline
Hauler & Lead/Iron & 35,000 & 400 & 70 & 40 & 5 \\
\hline
Gnasher & Lead/Iron & 100,000 & 900 & 100 & 40 & 20 \\
\hline
\end{tabular}
\end{center}

\begin{multicols*}{2}
\section{Spells}

\begin{spell}{Improved Illusory Script}
\Type{Second-Level Illusion}
\CastTime{10 minutes}
\Range{NA}
\Components{M (paper and ink)}
\Duration{Instantaneous} \\
When you first learn this spell, you generate two strings of magical runes: a \textit{private key} and a \textit{public key}.
These keys are magically linked.
Generally you will keep the private key in your mind, and write the public key down on a piece of paper.

You can give a copy of your public key to other creatures who know this spell, and receive their keys from them as well.
Once you have somebody else's key, you can send messages to them across any distance, including on other planes.
It takes approximately ten minutes to prepare a message for transmission, which you do by transcribing it onto a separate piece of parchment. 
Once transcribed, you the paper disappears in a puff of flame, and appears in front of the creature you wished to transmit to.
The transcribed paper can only be read by creatures who know the private key associated with the public key you transmitted to.
It takes approximately two minutes for creatures who know this key to decipher the message.
Before it is deciphered, it appears as a string of random letters, numbers, and punctuation.
\textit{Dispel Magic} does not remove this effect, nor does any source of true sight.

Creatures may tell others their private key, although this is highly dangerous, and should almost never be done.
Your private key can not be deciphered by mind-reading spells, and being told to reveal the key counts as damage for purposes of charm spells.

\end{spell}

\section{Misc. Items}

\subsubsection*{Elvesbane}
Elvesbane is a feared, legendary plant.
Said to be made from the essence of Lolth herself, it grows in the darkest, deepest places of the underdark.
When one soaks the root of an Elvesbane plant in boiling water, the resulting brew is a brutally efficient poison which works exclusively on elves.
With just one sip, an elf is doomed to die as all of their internal organs liquify---and no magic can save them.
Legend has it that drinking an entire pot would kill even Correllon him/herself.

\end{multicols*}


\end{table}
