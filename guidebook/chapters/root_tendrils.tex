\chapter{Geography}
Spirals is a world made up of Tendrils, strips of land that twist and turn throughout the sky.
These Tendrils form the base classification of geography in Spirals, similar to continents in our world.

The nine most important Tendrils are the \textit{root tendrils}, extremely large landmasses that begin at the core.
Most of the other tendrils branch off from one of these tendrils.
Each section in this chapter is dedicated to a specific root tendril, and the sub-sections within describe geographically interesting subtendrils or locations.

\begin{multicols}{2}
\section{The Great Flow}
The Great Flow is the name given by the citizens of Spirals to the largest water tendril, and the only root tendril created entirely out of water.
One of the nine root tendrils, the Great Flow starts at a giant portal and flows outwards from the center of Spirals, branching only occasionally.
The tendril is around two hundred miles wide and a perfect cylinder.
Unlike most of the other root tendrils, it does not travel in a straight line, opting instead to spiral and twist through the air as it travels.

The saltwater of the tendril hosts a variety of marine life, ranging from small fish near the edges to gargantuan sea monsters in the center.
Some sentient races fish the waters of the tendril, although the strong current makes this endeavor rather difficult.

\section{Ghiaccio}
Ghiaccio, by far, is the coldest of all the root tendrils.
From the second it leaves the center, the tube-shaped tendril is coated in ice and snow.
It would be completely abandoned, were it not for the great quantities of gold that appear in veins underneath its surface.
As a result of this potential wealth, both the outer and inner surfaces of the tendril are home to dwarven mines---and the occasional white dragon.

This tendril's light cycle takes much longer than normal, with nights lasting up to 1440 hours during the colder months, with extremely short days.
Even when the light is present, the tendril stays freezing, barely increasing in temperature.

\section{The Caldus Tendril}
The Caldus tendril is a vast desert landmass, and one of the nine root tendrils.
It is a flat tendril with only one navigable surface.
The tendril is home to massive sandstorms, and is generally inhospitable to life.
Even so, some believe that the tendril may be home to a lost civilization or some kind of ancient treasure.

The nights on Caldus last about an hour, with days as long as thirty hours.
Nights, as a result, provide very little shelter from the blistering heat.

The tendril would be completely abandoned were it not the most efficient way to get to Syngland.
Even though the planetoid is barely a hundred miles out from the Core, anybody daring to make the trip for trade purposes will no doubt need to stop at least once to quench their thirst and rest.
The most unusual city of Sethi provides this.

\subsection{Syngland}
Syngland is a planetoid located along the Caldus tendril, about five miles across.
It is well known for its Silk production, as well as for being one of the very few Tiefling settlements anywhere.
Unlike the Tendril it floats off, it is quite humid in climate, and several degrees cooler.
Most of the planetoid is a large silk farm, where magical silkworms spin the finest threads in all of Spirals.

Of course, since it's mainly populated by Tieflings, and a rumors of its true purpose abound.
Many think that the silk is actually provided by some demon source, and stories of small children being kidnapped, taken to Syngland, and forced to spin until they die are common in Nursery books.

\subsection{Sethi}
Sethi is one of the extremely few settlements on the Caldus tendril.
It is located about halfway between the core and the Syngland Planetoid.
The city is used almost exclusively by merchants delivering goods from Syngland, with the only permanent residents being a clan of Dragonborn.
This clan was blessed by a goddess long ago, who gave them a special portal that led within The Great Flow.
The high priest of the clan is able to control the rate of this at will, although the goddess specified that it should not be open too quickly.
The clan uses this water to provide an oasis along the harsh desert---for those with enough gold, of course. 


\section{The Kapatagan Tendril}
The Kapatagan Tendril, one of the nine root tendrils, is a rectangle around one hundred miles across on each side.
The corners of the tendril are all coated in water, forming four streams that flow outwards from the origin point.
The easy availability of water, combined with the mild climate, cause the tendril to be coated with plants and ripe for farming.
The typical settlers are human closer to the core, and halfling as one more farther away. 

At some points, the grass fields turn into thick forests.
Most attempt to get through those as fast as possible, to avoid death by the creatures that live within.

\subsection{Ero}
Sometimes known as the ``gutted tendril'', Ero one of the more interesting locations to be seen when traveling along Kapatagan.
With the exception of three large chains leading between it and its parent, Ero is connected to nothing at all, an attempt to remove a tendril to make room for trade routes gone awry.
Very little can be found on the surface of Ero, with the exception of small domes speckling the landscape that serve as infirmaries, seemingly pointless until one looks under the surface.
This is because the only city on Ero is entirely within the tendril; Lauriville.
Lauriville began as a small mining operation under Vivian Whitecliff and Emile Laurietti, and exploded into an elevated city of scaffolding, all within the confines of an ever-expanding cavern, almost entirely for miners to seek their fortune.
The magic holding the tendril together seems to be concentrated in the very center, a large core of stone, dirt and gravel going straight down the middle of the massive tunnel, as well as providing the base upon which the city rests. 

After Emile died under unclear and often disputed circumstances, Vivian took control of the growing facility, reforming it into a sort of business by which she allowed miners to work the caves and keep some of what they found, with a percent of the profits going back to her. 
Unfortunately, the seemingly endless resources of Ero showed signs of drying up some years later, and the former glory of the vast cavern died away. 
The city continues its operations with less fervor than it once had, but to an individual without a copper to their name it still poses a chance to strike it rich.

\subsection{Kheti}
Kheti is a major settlement on Kaptagan, focused mainly on farming.
Initially created by monks wishing to provide the residents of the Sacred Core with food, it quickly grew in size until it was a major farming community.

Kheti provides the entire city of The Core with food, ranging from grains to livestock.
The once peaceful city, however, grew a dark underbelly as it grew in size.
Illegal horse races take place in shady tracks, where criminal loan sharks are all too willing to give the desperately addicted money.
Some claim that ranch buildings are home not to cows, but slaves or drug factories.

\subsection{Crash}
Crash is an offshoot tendril from the Kapatagan tendril, about three hundred miles from its origin at the core.
The tendril is approximately one hundred miles long, and takes the form of a wide, thin rectangle, nine miles across on one pair of sides, and a quarter mile on the other.
Although once greatly populated, the crops were quickly overtaken by thick, thorny vines.
The settlers left, believing it cursed, and there no longer exists any major settlements on its surface.
Sentient creatures rarely venture inwards.

\section{The Dzan Tendril}
The Dzan Tendril is a wet, humid Tendril covered in jungle.
It has short days and nights, with each stage lasting around nine hours.

The tendril is moderately populated.
Its inhabitants include Elves, Dwarves, and Lizard-folk.
Settlements do not tend to last very long, however.
Typically, a city may survive for only a few decades before the residents decide to move to a more fertile spot.
This leaves the jungle full of ruins, causing the Dzan tendril to be a popular location for adventure stories.
\section{The Pacca Tendril}
The pacca Tendril is a cold, rainy tendril which meets the core near the Kpatagan tendril.
It is shaped like a large letter U, and curves around itself extremely slowly.
The large amount of rainfall cause its surface to be full of lakes and rivers, many of which provide enough fish to support a town or city.

While not freezing, the tendril's temperature are low enough that the people tend to wear heavy clothing.
Most inhabitants are from the heartier races as well, and elves are a rare sight on the Tendril.
The Tendril is best known for its bards, who follow the example set by the Tendril's most famous resident, David of the Burn, who used his music to defeat both a Lich and a dragon over the course of his life.

\section{Har}
Har is a mountainous, rocky Tendril.
Shaped roughly like a cylinder, Har is covered in steep valleys and tall mountains, making it nearly impossible to navigate by land.
Instead, special air trams, which run along magical cables, provide the primary form of transport.
These trams are not under any unified control, but instead make an ad-hoc network across different countries, making them slightly less than reliable.
\todo{Expand this section}

\section{Tyling}
Tyling is a humid, wet Tendril.
Known for its many bogs and marshes, it is an easy source of farmland for certain crops, and also exports a large amount of coil, oil, and peat.



\todo{Expand this section}

\section{Mozeya}
Mozeya is a triangular tendril with ninety-mile sides.
It is mostly covered in a forest with a fairly sparse canopy, creating a savannah-like environment where large wildlife thrive.
\todo{Expand this section}

\section{The Missing Tendril}
 The orientation of the root tendrils suggests that there should be one more, between the Kapatagan and Caldus Tendrils.
No such Tendril, however, has ever been known to exist.

Wizards have long found trace magical energies where this Tendril would have met the core, leading many to believe that there used to be a Tendril in this location.
This Tendril is commonly referred to as ``The Missing Tendril'' or ``The Tenth,'' and is the subject of conspiracies, ghost stories, and religious tales the world over.

\end{multicols}
