\chapter{Tendrils}


\begin{multicols*}{2}
\section{Physics}
The titular spirals of Spirals are great masses of material floating in a magical aether.
They are made up of many materials, ranging from solid rock to loose dirt to water.
Some tendrils are even made entirely of air or other gasses.

The tendrils have gravity, but it works a bit differently than it does in real life.
The gravity of a tendril is primarily a function of the magic keeping it up.
As a result, some tendrils can be tubes as little as ten feet around, but still have gravity to hold large objects or creatures safely on their surface.
Tendrils which are close to each other may have gravitational fields so small that one can hop from one to the other, naturally finding their feet pulled to the floor as they transition.

Tendrils have a wide range of sizes. 
Some of the smaller ones can be thin pillars of earth that a typical human could wrap their hand around, while others can be so thick they take years to encircle.

\subsection*{Lighting}
Tendrils experience a day and night cycle, just like in reality.
Unlike in reality, however, the light generally does not have a defined source.
Lighting and heat are a function of the same magic that holds up the tendrils, and patches of day move across them in strips several thousand miles wide.
Each strip takes around sixteen hours to move across a tendril during the warmer months, with a "night" lasting around six hours.
During the colder months, the night can last up to twelve hours, with day taking as few as five.
Different tendrils may have different timings.

\subsection*{General Structure}
The tendrils are not a completely unorganized mess.
There are six main tendrils, the arteries of the world. They start at a central location, from the core and grow outwards, growing in width as they travel apart.

Those tendrils eventually branch off into many more, around a hundred miles outside of the city.  These tendrils continue to branch, some of them growing side branches along the way.

This is not to say that the tendrils are a perfect mesh network.
Some are not attached to anything, floating alone in space.
Some twist back on themselves, or spin off in random fractal directions.
Some may even take the form of floating spheres or cubes.


\subsection*{Tendril Collapse}
The tendrils of Spirals are, in the general case, incredibly stable.
Physical forces seem to have no effect on the actual position of the landmasses, and even the most powerful of explosions will, at most, leave a large crater or obliterate a chunk of a thinner spiral, leaving both ends intact.
Under rare circumstances, however, a tendril can collapse completely.
Typical causes for this are:
\begin{description}
\item[Intentional Destruction:] groups of extremely powerful wizards may be able to collapse a tendril entirely by unwinding the magical framework in which it sits. This is occasionally done by governments to non-inhabited tendrils, in order to make trade routes easier.
\item[Extreme Magical Disturbance:] certain magical acts may cause the magic holding a tendril up to unravel on its own, destroying a spiral.
\item[Acts of the gods:] the more powerful gods may chose to collapse a tendril to punish mortals. It should be noted, however, that collapsing a tendril with people on it is one of the most extreme actions a god can take. Even chaotic evil gods may think twice before doing so.
\item[Natural Decay:] some tendrils simply lose power over time, and will collapse on their own. This process, however, takes quite a long time, and those who are magically sensitive will easily be able to detect what is happening. 
\end{description}

\subsubsection{Adventure Idea}
Tendril collapse can provide an interesting plot hook for your own adventures.
Perhaps a crazed villain decides to destroy a tendril for his own material gain, or simply because he does not like it.
Perhaps the players must convince a group of wizards to help them collapse a spiral that houses an evil dragon or tarrasque, before it wakes up.
Perhaps a spiral has suddenly collapsed, killing thousands and leaving the survivors pointing fingers at their rival groups, looking for somebody to blame.


\section{Creating a Tendril}
The world of Spirals is home to uncountable tendrils, making their way across the stars.
Some of the major ones are described in this document.
However, the situation may arise in your campaigns where you wish to throw your players onto a brand-new tendril, that you created, or when you want a unique Tendril for a character backstory.
Here, we will provide some general guidelines for doing so, as well as roll tables to help you if you get suck (or if you just like the element of randomness).

\subsubsection*{Few Limitations}
The nature of tendrils gives you extreme freedom as a DM.
In fact, a tendril is able to have essentially any shape, size, environment, or other characteristic that you desire.

As such, the limitations mainly exist in terms of \textit{narrative consistency}.
A tendril should not logically break the society of Spirals.
For example, a tendril made of solid gold that's thousands of miles long would reduce the value of gold to near-worthlessness, due to the laws of supply and demand.
Putting such a tendril in your campaign may not be a good idea---unless you can justify it.
For example, perhaps the surface of the tendril is completely impenetrable, mocking those who attempt to mine it.
Maybe it is extremely far away from any settled area, accessible only via a long and near-suicidal journey through a tendril that alternates between bitter ice and scorching desert, and your players are simply the first to reach it.
Perhaps everybody knows it exists, but it is floating by itself far from any other landmass, through a vacuum that is unaccessible for even flying creatures, with an aura around it that prevents magical teleportation.\footnote{Such a tendril is almost guaranteed to be iresistable to your players, likely eventually resulting in a spaceship mission to acquire its wealth. Of course, a space race campaign seems to have a lot of potential, so perhaps that's a good thing.}

In truth, almost nothing is completely off-limits.
If you have an idea for an especially unique Tendril, just be sure to think through its logical consequences.
You may even want to run your idea past others, to get a second opinion on the matter.

\subsection{Shape}
Tendrils come in many shapes.
We list some here.

\begin{rolltable}{Tendril Shapes (d12)}
\item[1-2] Cylinder
\item[3-4] Rectangular Prism
\item[5-6] Hollow cylinder, with walkable inside surface
\item[7-8] Half pipe
\item [9-10] Flat surface
\item[11] Planetoid
\item[12] Floating chunks of land
\end{rolltable}

\subsection{Climate}
Tendrils vary widely in climate, from dense jungle to arid desert.
Some are even composed entirely of water.
Among those that do stuff, things are done. Yes.

\begin{rolltable}[0.5\textheight]{Tendril Climates (d10)}
\item[1] Underwater
\item[2] Jungle
\item[3] Desert
\item[4] Forest
\item[5]  Swamp
\item[6] Tundra
\item[7] Plains
\item[8] Ice
\item[9] Rocky
\item[10] Volcanic
\end{rolltable}


\end{multicols*}