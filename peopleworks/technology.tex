\chapter{Technology}
The world of Spirals requires some special technology, mostly to help in transportation.
Interesting locations may be hundreds of miles away from each other, and nobody wants to play a campaign where the vast majority of time is spent in transit.


\begin{multicols}{2}
\section{Landcycles}
The primary form of personal transportation in Spirals is the landcycle.
These vehicles use the magic that holds the spirals up to travel long distances at extreme speeds.
Landcycles are a common piece of technology, and can generally be acquired cheaply.
They can be made from various materials, ranging from wood to metal to strange matter.
However, they have a few important catches:
\begin{description}
\item[Innate Downtime:] when a cycle travels for a certain amount of time, it must rest before it can travel further. 
This resting period lasts anywhere from a few hours to an entire week at a time.
If they run out of power in an undesirable area, the rider has a problem on his or her hands.
The travel time of a lightcycle can be increased by feeding it a magically-charged gemstone, with more expensive gemstones having a greater effect.
\item[Slow to Start:] A landcycle can take quite a while to start moving.
The cycles require a rider to be on them at all times while warming up, and if a rider dismounts for more than sixty seconds the countdown begins anew.
This makes them rather impractical to use as escape vehicles.
\item[Limited Carrying Capacity:] Landcycles can carry only approximately four hundred pounds.
If one wishes to transport a larger weight, they need to look for other options.
\end{description}
\end{multicols}
\begin{table}[hb]
\caption{Types of Landcycle}
\begin{center}
\begin{tabular}{|c | c | c | c | p{0.07\linewidth} | p{0.1\linewidth} | p{0.1\linewidth}| }
\hline
Name & Material & Cost (GP) & Range (mi) & Speed (MPH) & Downtime (hours) & Startup Time (minutes) \\
\hline
Jaunter & Wood & 100 & 200 & 45 &10 & 15 \\
\hline
Cityrunner & Wood/Metal & 200 & 500 & 60 & 7 & 15 \\
\hline
Trekker & Wood & 600 & 800 & 60 & 10 & 20 \\
\hline
Harlot & Iron & 1,000 & 1,000 & 75 & 15 & 10 \\
\hline
Blackbird & Steel & 10,000 & 2500 & 250 & 10 & 5 \\
\hline

\end{tabular}
\end{center}
\end{table}

\begin{multicols}{2}
\section{Landchewers}
A slightly less common (and much more expensive) form of transportation is the Landchewer.
They can hold much greater weights---up to several tons---but have their own unique disadvantages.
\begin{description}
\item[Preset Route:] A Landchewer cannot be manually piloted.
Once its route is set, it will follow it until it reaches its destination.
The more expensive models do stop moving if they detect significant changes in the terrain, but none are able to be steered around new obstacles.
\item[Extremely Slow to Start:] A Landchewer takes exactly 50\% of the time it will be in transit to plan out its route.
While they are faster than landbikes to compensate, this caveat makes them worthless unless one knows exactly where they need to be far ahead of time.
\end{description}
\end{multicols}
\begin{table}[hb]
\caption{Types of Landchewer}
\begin{center}
\begin{tabular}{|c|c|c|c|c|p{0.1\linewidth}|p{0.1\linewidth}|}
\hline
Name & Material & Cost (GP) & Range (mi) & Speed (MPH) & Downtime (Hours) & Max Weight (tons) \\
\hline
Tortoise & Iron & 10,000 & 200 & 45 & 24 & 0.75 \\
\hline
Ripper & Iron & 20,000 & 250 & 60 & 24 & 2.0 \\
\hline
Hauler & Lead/Iron & 35,000 & 400 & 70 & 40 & 5 \\
\hline
Gnasher & Lead/Iron & 100,000 & 900 & 100 & 40 & 20 \\
\hline
\end{tabular}
\end{center}

\end{table}
