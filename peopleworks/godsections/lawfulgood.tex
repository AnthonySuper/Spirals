\subsection*{Bahamut}
\begin{goddesc}
\Level{Major}
\Domains{Knowledge}
\Symbol{The head of a dragon, in silver}
\end{goddesc}
Bahamut, the lord of good dragons, is a powerful force for good in the world of Spirals.
An ancient and powerful God, he has been alive for as long as mortals have recorded history, and still interacts with the world on a daily basis.
Unlike his Sister, he cares for all creatures, great and small, and fights daily so they may live their lives in a world filled with justice and peace.

\subsubsection*{Personality}
Bahamut is a kind soul with a firm personality.
It is rare to see a smile grace his features, and he can be stern in his dealing with his clerics.
This is not out of any distaste for mortals---indeed, he cares for the lives of those beneath him more than most of his draconic worshippers, and is far more active in their affairs than he is obligated to be.
Instead, his emotionless composure is borne of the grave nature of his task: evil is everywhere, and fighting it is a constant, endless struggle.

In certain forms, the dragon's personality may be altered.
When traveling in the disguise of an old man, for example, he can seem almost mad, laughing for no discernible reason and sprouting vague prophecies.
Some see this is a sign of inner turmoil, while others believe it to merely be a form of stress-relief.

\subsubsection*{Physical Appearance}
Bahamut normally takes the form of a colossally large dragon, majestic and beautiful in the same way as a piece of great architecture.
His scales shimmer and glow as if light by moonlight, and his dark grey eyes seem to contain storms.

Bahamut occasionally travels in other forms as well.
Most notably, he often mingles among mortals in the form of a human, appearing as either an insane old man wearing tattered robes or a handsome noble in a simple cloak.


\subsubsection*{World Temple Room}
Bahamut's complex in the world temple is impressive, so much so that many clerics who do not worship the platinum dragon make stops to visit it when on official business for their own guards.
The marble walls are coated in artwork, and every room has at least one sculpture.
A grand red carpet leads to the throne room of the god himself, where he sits surrounded by yet more art, as well as a carefully-organized collection of history books.
A door behind his throne leads to the room where he keeps his own horde, which consists largely of enchanted weapons, along with coins and other treasure.

Bahamut's temple is not only inhabited by the god himself, but also his most trusted servants.
A variety of ancient clerics live in the temple, performing divine research, surveying the outside world, and occasionally even answering prayers on their master's behalf.
These clerics have their own rooms, as well as common working areas that they all share.

The vast majority of those invited to live with the god are dragons, but Bahamut will occasionally grace a member lesser race this honor if they perform great feats in his name.
For a cleric loyal to the platinum god, this is an honor unimaginable.

\subsection*{Byr'nke the Wise}
\begin{goddesc}
\Level{Intermediate}
\Domains{Knowledge}
\Symbol{Three gold coins surrounding an eye}
\end{goddesc}
God of money and monetary policy, Byr'nke the wise is a respected, if little-known, deity.
Whenever wise and prudent treasurers discuss their currency, his presence is felt, guiding their gentle hands.

\subsubsection*{Personality}
Byr'nke is a kind, if somewhat distant, god.
His clerics help in just financial institutions the world over, guiding their decisions and easing their extreme stress.
Some even rumor that Byr'nke's avatar will occasionally instruction students on the subject of economics at several colleges throughout the world.

\subsubsection*{Physical Appearance}
Byr'nke takes the form of a humanoid in banker's robes.
His exact species changes depending on the audience: elven when among elves, human amidst humans, and dwarven in the presence of dwares.
Some have even claimed that he has draconic or demonic forms, although they are rarely used if they exist.
The only common feature among his forms is his facial hair, a short goatee.

\subsubsection*{World Temple Room}
Byr'nke's room is extremely small.
it consists of several offices, occupied by his clerics, and one head office, occupied by himself.
There is also a boardroom where he meets with financial leaders or other gods.
The table in this room was donated to him as a gift from Bahamut, and is quite impressively made of solid gold.

\subsection*{Tecumtaz}
\begin{goddesc}
\Level{Major}
\Domains{War}
\Symbol{A flaming hammer}
\end{goddesc}
Tecumtaz is the god of just war, and perhaps one of the strangest gods in Spirals.
Wherever mortal fight tyranny and oppression, he is there, but he is not a god of kindness, and is barely one of restraint.

\subsubsection*{Personality}
Tecumtaz hates war.
This simple fact burns throughout his personality and actions.
Unlike other deities of war, who respect great warriors and bless mighty conquerors, the mightiest soldier in Tecumtaz's eyes is the one who ends the conflict the most quickly and justly, honor be dammed.
This hatred seeps into all other parts of his personality.
The face he presents to those who righteously end conflicts is one of supreme kindness, almost familial in nature.
Regardless of their status, he respects those people above all others.

To those fools who start senseless conflict, however, Tecumtaz presents a very different face, one full of terror and fury beyond all mortal comprehension.
This includes many other gods, even ones who supposedly share his alignment and domain.

If a cleric of Tecumtaz is to join a war, one can expect absolute chaos---scorched-earth tactics, destruction of civilian property, and large-scale destruction, although they will firmly prevent the deaths of the innocent.
One will also expect that, for better or for worse, the war will soon be over.

\subsubsection*{Physical Appearance} 
Tecumtaz most often takes the form of a man in an officer's uniform.
When enraged, his eyes glow with terrible fire.

\subsubsection*{World Temple Room}
Tecumtaz's room in the world temple takes the form of a military encampment at dusk, a vast field filled with several tents.
The air inside is cool and damp.
Tecumtaz himself takes position within the officer's tent, which is only slightly bigger than the rest.
Any clerics who happen to be residing with him stay in other tents.