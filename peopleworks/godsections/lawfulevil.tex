\subsection*{Tiamat}
\index{Gods!Tiamat}
\begin{goddesc}
\Level{Major}
\Domains{Knowledge}
\Symbol{Five Dragon Heads in Profile}
\end{goddesc}

Tiamat is the dragon goddess of greed and wealth, as well as the mother of all dragons.
You most likely recognize her---she's present in many D\&D settings, although her personality in Spirals is  altered.
Tiamat, like Bahamut, is an extremely old God, and has spent most of her long life becoming impossibly, ludicrously wealthy.
Tiamat sees the acquisition of wealth as a sacred duty, which dragons have a responsibility to exercise.

\subsubsection*{Personality}
Tiamat is a pragmatic deity, perhaps more so than any other draconic diety, or even most dragons.
She looks at her wealth over the long term, and is more than willing to part with small parts of her horde to increase its size later on.
It is not uncommon for her to loan her clerics rare magical items, scrolls, or other useful items.
She has even been known to give those exceptionally loyal to her small gifts, to incentivize similar behavior in others.

Tiamat views her pride as another component of her wealth, but her pride is not fragile.
The insults of a small creature far away are almost certainly not worth the time to incinerate them, and she respects the few who have more power than her.

\subsubsection*{Physical appearance}
Tiamat typically takes the form of a dragon with five heads, one for each chromatic color.
She occasionally also takes the form of a beautiful young woman, dressed in stupendously valuable clothing.

\subsubsection*{World Temple Room}
Tiamat's room in the world temple takes the form of an elegant palace.
The walls are lined with Platinum, carved into intricate murals, and decorated with fine art.

Behind her throne, in a room only she may access, Tiamat keeps the bulk of her horde.
Few have set eyes on it, and fewer have spoken of it afterwards.
What can be said for certain is that it is staggeringly large, and contains items the likes of which can hardly be described.

\subsection*{Bane}
\index{Gods!Bane}
\begin{goddesc}
\Level{Major}
\Domains{Trickery}
\Symbol{A cloud of smoke}
\end{goddesc}
Bane, the god of fear and Tyranny, is one of the most hated (and most worshipped) gods in the Spirals pantheon.

\subsubsection*{Personality}
Bane's personality is surprisingly charismatic.
He is confident, oddly funny, and even cordial.
Even those who hate what he stands for must admit that he is quite the charmer.
Tyranny, of course, is best executed by a leader who is at least somewhat liked, and fear is most effective when one's guard is down.

\subsubsection*{Physical Appearance}
Bane normally takes the shape of a handsome man, in his mid-twenties, wearing a black robe.
When he wishes to personally spread fear, he takes on an amorphous form, like black smoke.
This form quickly morphs to depict what the viewer fears most.
When in this form, all that view him must succeed on a DC 30 wisdom saving throw, or become frightened for 1d8 days.

\subsubsection*{World Temple Room}
Bane's world temple complex is extremely sparse.
It consists mainly of his throne and a few portraits of himself.
All else is located behind a small door, which has his symbol on it.
Many have been brought behind the door---none returned in any state to tell the outside world what they saw.