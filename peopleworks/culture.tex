\chapter{Culture}
The world of Spirals, like all other worlds, has its own unique culture.
This will probably influence your character creation.

This chapter contains both general and specific cultural descriptions.
The more specific descriptions are provided to assist you in designing your own unique subcultures.

\begin{multicols}{2}
\section{Religion}
The various faiths and cults in the world of Spirals are, like everything else, effected by its geography.
Various tendrils are considered holy to different deities, and temples are somewhat more commonplace.

In Spirals, deities are significantly less ambiguous than they are in other settings.
Since each has a space in the world temple, it is relatively easy to enumerate them.
Different groups will obviously worship and pay homage to different sets of deities, of course, but most are aware that their gods are not the only gods.

\subsection{Joint Temples}
Temples catering to two or more gods are somewhat more common in Spirals than in other settings.
Modeled after the World Temple, these joint temples typically consist of small worship spaces for each god, and one large general room that is used for more major ceremonies.
Typically the gods worshipped by these joint temples share some major feature, such as domain or alignment.

\begin{adventureidea}{Civil Dispute}
Sometimes an event at a joint temple can cause tension between the two groups of worshippers that use it.
Perhaps the clerics of Correllon are missing a sacred bow, and begin to blame the followers of Tecumtaz for taking it.
Maybe the clerics are having trouble contacting their god, and are beginning to suspect the others of committing some heresy to cause it.
Whatever the case, a conflict between members of a joint temple can be a great dispute for your players to solve.
\end{adventureidea}

\subsection{The World Temple}
The most major difference comes from the temple within the World Temple.
The temple allows clerics and worshippers to have a much more direct location with their gods.
If one is particularly troubled, they may make a pilgrimage to the temple, to ask their god what to do.
It is unlikely, of course, they that will get to meet the god themselves, but praying near their physical presence tends to be more effective---and most of the world temples are filled with helpful clerics.

\section{Magic}
Spirals is a deeply magical land.
Magic holds up the Tendrils, lights and warms them, and is ultimately the source of all life.
People are still skeptical of magicians, but not magic itself---Wizards are strange not because they \textit{have magic}, but of the way in which they are able to \textit{control} it.

\subsection{Magical Technology}
Most people in Spirals will interact with low-level magical objects on a near-daily basis.
Farmers, for example, often use Landchewers to drag plows in their fields, and a seamstress may use a magical spinner to turn her wheel.
As a result of the plentiful magic, Spirals is significantly more industrialized and metropolitan than many D\&D settings.
That's not to say that there aren't remote farming villages or unexplored caverns---across all the tendrils, almost everything is bound to exist in some capacity.

If you wish, you can keep in mind this difference in technology when crafting a character backstory.
The farmhand child dreaming of the outside world, for example, may have spent his days working with finicky equipment more often than his hands, and spent his nights reading a serialized adventure magazine by candlelight.
\section{Magic Users}
As outlined in the \hyperref[chapter:magic]{chapter on magic}, Magic behaves differently in Spirals than in many other worlds.
This, naturally, comes with some different cultural attitudes towards those who use it.

\subsection{Wizards}
Wizarding in the World of Spirals has a unique place, owing to the unique role of magic in the world.
Magic in Spirals is more prominent in Spirals than it is in other worlds, but this prominence adds danger.
The risk inherit in spellcasting makes wizarding a very mentally taxing job, both because of the risk and the mental fortitude required to overcome it.
Wizards can be good-natured, but the stress inherit in their profession makes many irritable. 

In order to reduce this risk and manage the stress of using magic, wizards in Spirals are carefully trained.
The process of becoming a wizard is extremely difficult, requiring nearly a decade of schooling and mentorship.
Wizards jealously guard major spells, and can be extremely competitive in their knowledge of the arcane, but they have sympathy for each other as well.
Wizards will generally be friendly with each other, assuming they have no major differences in worldview or goals, and will provide help to each other in times of need.

In their old age, wizards look forward towards their legacies, and backwards at their origins.
Older wizards who are not consumed with research will often seek to train apprentices or students, and many of the most powerful wizards will eventually teach at the College De Arcana.

\subsubsection{Interactions with Society}
Wizards are generally respected, if slightly feared.
For most citizens in Spirals, they will see very few wizards in their lives, although they may use tools crafted by them on a daily basis.
Wizards are viewed as learned beings, but the risks involved in using magic lead many to view them as having a certain undercurrent of imprudence or recklessness.
Many would eat dinner with them in a public place, but few would invite them into their home.

\subsubsection{Interactions with Warlocks}
Many wizards view the arrangement between a Warlock and his Patron as a kind of cheating.
Wizards are distrustful of Warlocks, and generally avoid them.

Some of the more ambitious wizards are actively performing research into Warlock patronage, with the end goal of exploiting the methods patrons use to control magic for their own gain.
Patrons typically distrust these wizards.

\subsubsection{Interactions with Sorcerers}
Wizards generally view sorcerers with pity, mixed with suspicion.
While they won't say it in polite company, Wizards often refer to them as``Powderkegs", ``Flares", or ``Kindling," although this is done more as sympathetic gallows humor than anything else.

Long ago, wizards learned that the magic of sorcery could not be controlled via conventional wizarding means: the sorcerers who tried all died in spectacularly violent ways.
Since then, many wizards have sought to study sorcery, in the hopes of one day understanding its causes.
The end goal varies from researcher to researcher: some wish to harness its power, some wish simply to understand, and others wish find some way to magically reduce the mortality rate.
As of yet, however, sorcery remains a mystery.

\subsection{Warlocks}

Warlocks gain their power not from careful study, but from a patron.
They rely on both their own knowledge of magic, and their patrons, to cast spells, although the patron has significantly more control.
Their patrons chose them well, and they are generally safe from the stress of arcane spellcasting---although their patron may cause them stress of a different variety.

\subsection{Sorcerers}
Sorcerers derive their magic from their own nature.
For reasons unknown, magic is especially active around them, pressing in on their minds, visibly moving through their bodies.

Sorcerers are more commonplace in Spirals than in most other worlds.
Unfortunately, however, they rarely survive until adulthood.
For most, the magic is too much for them, and it consumes them early in their development---most often, this consumption is literal, as they burst into flames.
Those who survive until adulthood are reasonably stable, however, it is not unheard of for them to lost control and burst into flames as well.
The author of this handbook does not recommend that you have this happen to any of your player characters.
\footnote{There are exceptions to this, for example, rolling 10 or more critfails consecutively, or when you desperately need to get rid of a player with a particularly foul personality or odor.}

Those who do survive into adulthood are viewed with some respect for the mental fortitude that allowed them to do so, but also with unease that they may lose control at any moment.

\subsubsection{Interactions with Wizards}
Sorcerers generally look at wizards in one of two ways.
The first is as a kindred spirit, a fellow user of magic, even if their methods are different.
The other is as insane men who willingly inflict themselves with a curse the sorcerer had no control over.
Sometimes it is a strange mixture of the two, which can result in rather passive-aggressive behavior mixed with admiration.

\subsubsection{Interactions with Warlocks}
Sorcerers are skeptical at best of Warlock packs.
They know better than anybody what magic can do, and how difficult it is to control---which is why they're inclined to disbelieve a man in dark robes who claims that he is in an "equal contract" with a being so magically powerful.


\end{multicols}