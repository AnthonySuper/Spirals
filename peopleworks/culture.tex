\chapter{Culture}
The world of Spirals, like all other worlds, has its own unique culture.
This will probably influence your character creation.

This chapter contains both general and specific cultural descriptions.
The more specific descriptions are provided to assist you in designing your own unique subcultures.

\begin{multicols}{2}
\section{Religion}
The various faiths and cults in the world of Spirals are, like everything else, effected by its geography.
Various tendrils are considered holy to different deities, depending on their composition.
Most of these relationships, like the Frozen tendril being a holy place of Auril, are obvious.

\subsection{The Core}
The most major difference comes from the temple within the Core. 
The Core's temple, commonly reffered to as "The world temple," is a rather unique place in that its continued peace and harmony is beneficial to all gods with any sense, regardless of how evil they are.
Some chaotic evil deities may be foolish enough to try to disturb the peace, or to get their worshippers to do so, but the other deities tend to put them in their place rather quickly.

This continued peace, however, comes at the rather sticky cost of putting a large number of opposing cults in close proximity.
Worshippers of Tiamat and Bahamut may meet on a daily basis in the temple, as do those of Torm and Bane.
This forced interaction means that rival groups tend to be in even greater conflict than they normally would, and religious wars are slightly more common.
On the flip side, this exchange does mean that some worshippers have greater opportunities to switch sides, with little their deity or cult can do about it---as long as they don't leave the world temple, that is.

The world temple provides one other religious feature, in the form of its center chamber.
Although only a very privileged few are allowed access---indeed, some gods themselves have not been inside it---most creatures know what exists inside.
The center chamber is a perfect cube of pure white marble, exactly three hundred meters on each side.
It is completely devoid of any distinctive features, except for the object at its center.
There lies what would be known on earth as a Mobius Strip, also made of marble, floating perfectly and rotating at exactly one revolution per second.

All of the more powerful (and some would say arrogant) deities claim that this symbol is theirs, put there by the universe to signify their ultimate righteousness or power.
As a result, this symbol can be seen in almost every temple, in many households, and even scratched underneath barstools or in latrines.

\section{Magic Itself}
Owing partially to the dangerous nature of casting spells, many wizards employ enchantments as opposed to directly casting spells.
The huge amounts of ambient magic present throughout Spirals allows simple spells which require only power to function to thrive.
Mechanically simple motion or heat charms have the most practical uses, and are by far the most commonplace.
Most notably, they are used in Landcycles and Landcrawlers, which you can read about in the technology section of this document.

Spells which require higher levels of precision are much harder to contain within an object.
If one wished to create a sword which cut through enemies more efficiently, for example, they would need to take care to temper the enchantment so it draws only the required amount of energy from the surrounding area.
Failure, of course, would result in the sword exploding violently the moment the enchantment was cast.
As a result, enchanted weapons and items are rarer than in other settings, as they are more difficult to make.

It should be noted, however, that the high levels of ambient magic increases the overall potential for enchanted items.
Few, if any, wizards have ever lived who were skilled enough to take advantage of this potential, but, if they could, the results could be wondrous (or devastating) indeed.

\section{Magic Users}
As outlined in the \hyperref[chapter:magic]{chapter on magic}, Magic behaves differently in Spirals than in many other worlds.
This, naturally, comes with some different cultural attitudes towards those who use it.

\subsection{Wizards}
Wizarding in the World of Spirals has a unique place, owing to the unique role of magic in the world.
Magic in Spirals is more prominent in Spirals than it is in other worlds, but this prominence adds danger.
The risk inherit in spellcasting makes wizarding a very mentally taxing job, both because of the risk and the mental fortitude required to overcome it.
Wizards can be good-natured, but the stress inherit in their profession makes many irritable. 

In order to reduce this risk and manage the stress of using magic, wizards in Spirals are carefully trained.
The process of becoming a wizard is extremely difficult, requiring nearly a decade of schooling and mentorship.
Wizards jealously guard major spells, and can be extremely competitive in their knowledge of the arcane, but they have sympathy for each other as well.
Wizards will generally be friendly with each other, assuming they have no major differences in worldview or goals, and will provide help to each other in times of need.

In their old age, wizards look forward towards their legacies, and backwards at their origins.
Older wizards who are not consumed with research will often seek to train apprentices or students, and many of the most powerful wizards will eventually teach at the College De Arcana.

\subsubsection{Interactions with Society}
Wizards are generally respected, if slightly feared.
For most citizens in Spirals, they will see very few wizards in their lives, although they may use tools crafted by them on a daily basis.
Wizards are viewed as learned beings, but the risks involved in using magic lead many to view them as having a certain undercurrent of imprudence or recklessness.
Many would eat dinner with them in a public place, but few would invite them into their home.

\subsection{Warlocks}

Warlocks gain their power not from careful study, but from a patron.
They rely on both their own knowledge of magic, and their patrons, to cast spells, although the patron has significently more control.
Their patrons chose them well, and they are generally safe from the stress of arcane spellcasting---although their patron may cause them stress of a different variety. 

\subsubsection{Interactions with Wizards}
Many wizards view the arrangement between a Warlock and his Patron as a kind of cheating.
Wizards are distrustful of Warlocks, and generally avoid them.

\subsection{Sorcerers}
Sorcerers derive their magic from their own nature.
For reasons unknown, magic is especially active around them, pressing in on their minds, visibly moving through their bodies.

Sorcerers are more commonplace in Spirals than in most other worlds.
Unfortunately, however, they rarely survive until adulthood.
For most, the magic is too much for them, and it consumes them early in childhood.
Those who survive until adulthood are reasonably stable, however, it is not unheard of for them to lost control and burst into flames as well.
The author of this handbook does not recommend that you have this happen to any of your player characters.
\footnote{There are exceptions to this, for example, rolling 10 or more critfails consecutively, or when you desperately need to get rid of a player.}



\subsubsection{Interactions with Wizards}
Wizards generally view sorcerers with pity, mixed with suspicion.
While they won't say it in polite company, Wizards often refer to them as "Powderkegs", "Flares", or "Kindling," although this is done more as sympathetic gallows humor than anything else.

Long ago, wizards learned that the magic of sorcery could not be controlled via conventional wizarding means: the sorcerers who tried all died in spectacularly violent ways.
Since then, many wizards have sought to study sorcery, in the hopes of one day understanding its causes.
The end goal varies from researcher to researcher: some wish to harness its power, some wish simply to understand, and others wish find some way to magically reduce the mortality rate.
As of yet, however, sorcery remains a mystery.





\end{multicols}